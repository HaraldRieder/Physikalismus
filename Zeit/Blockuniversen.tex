\documentclass[12pt]{article}
\usepackage[margin=2cm, bindingoffset=0cm]{geometry}
%\usepackage[german]{babel} 
\usepackage[ngerman]{babel} %sudo apt-get install texlive-lang-german
\usepackage{hyperref} % web links etc.
\usepackage[parfill]{parskip}
\usepackage[utf8]{inputenc}
\usepackage[dvipsnames]{xcolor}
\usepackage{tcolorbox}
\usepackage{helvet} 
\usepackage{framed}
\usepackage{anyfontsize}
\usepackage{csquotes}
\usepackage{mathrsfs}
\usepackage{physics}
\setcounter{MaxMatrixCols}{16}
\usepackage{amssymb}
\usepackage{MnSymbol}
\MakeOuterQuote{"}
\definecolor{quotecolor}{rgb}{0.8,0.9,1}
\renewcommand{\familydefault}{\sfdefault} 
\setlength\parindent{0pt}
\tcbset{boxrule=0pt,colback=quotecolor,arc=5pt,auto outer arc,left=5pt,right=5pt,boxsep=5pt}
%  width=0.9\textwidth,

\setcounter{secnumdepth}{-1}
\begin{document}
\title{\fontsize{25}{25}\selectfont \textbf{Eine Abfolge von Blockuniversen?}}
\author{Harald Rieder}
\date{\today}
\maketitle

%\begin{abstract}

%\end{abstract}

\tableofcontents

\section{Motivation}

Das Messproblem der Quantenmechanik entsteht aus der Vorstellung einer stetigen Entwicklung von mathematischen Größen in der als absolut angesehen gemeinsamen Zeit in Verbindung mit der Erfahrung, dass Kenntnis über das durch die mathematischen Größen beschriebene Quantending nur dann erlangt werden kann, wenn sich dabei dessen Zustand unstetig ändert. Dadurch sind zu jedem Zeitpunkt 2 verschiedene sich daran anschließende Arten von Zukünften möglich und niemand kann sagen, wann oder warum die eine oder die andere gewählt wird. 

Dieses Paradoxon hat schon die besten Köpfe herausgefordert und der ein oder andere dachte sich, eine Lösung zu haben. Mitunter taucht die Denke auf, dass verschiedene Sichten auf die Quantenmechanik möglich seien, so genannte "Interpretationen der Quantenmechanik", die am Ende doch nur eine Frage des Geschmacks seien. Wir werden hier diesen Weg nicht gehen. Statt dessen soll das Paradoxon dadurch aufgelöst werden, dass zu jedem Zeitpunkt nur eine Art von Zukunft möglich ist. Anstatt an der Vorstellung eines Zeitparameters, von dem Geschehen stetig abhängt, festzuhalten, werfen wir diese Vorstellung komplett über Bord. Es gibt einige gute Gründe, es auf diese Art zu versuchen:

\begin{enumerate}
    \item Unterhalb der Planck-Zeit kann es keine stetige Zeitentwicklung mehr geben.
    \item Eine Propagation des Geschehens durch lineare Operatoren, wie sie für die sich stetig anschließende Zukunft im Modell geschieht, treibt das Weltgeschehen eben \emph{linear} weiter. Man kann schwerlich glauben, dass dadurch irgend etwas wie Lebendigkeit erfolgreich modelliert werden kann. Aus der klassischen Chaos-Theorie haben wir gelernt, dass nur eine nichtlineare Dynamik zu etwas führt, dass uns an Leben erinnert. Die unstetig sich anschließende Zukunft liefert uns dagegen eine nichtlineare Dynamik.
    \item Um Kenntnis über ein Quantending zu erlangen ist ein Verlassen der stetigen Entwicklung notwendig. Die Dekohärenztheorie kann erklären, wie Information aus einem Quantending in seine Umgebung fließt unter der Annahme einer gemeinsamen stetigen Entwicklung. Doch wenn Kenntnis über Quantending und/oder Umgebung erlangt werden soll, ist ein unstetiger Schritt notwendig. Das \emph{problem of outcomes} wird durch die Dekoränztheorie nicht gelöst, und das ist das eigentliche Paradoxon.
    \item Eine komplett neue Dynamik könnte erfunden werden. Das erscheint aber schwieriger, als einfach die Hälfte der Dynamik wegzuwerfen.
\end{enumerate}
Die Quantenmechanik liefert uns statistische Größen: Wahrscheinlichkeiten für Ereignisse. Wir haben gelernt, wie man Empfindungen mit den Ereignissen zu verknüpfen hat, die das mathematische Modell hergibt. Dabei hat sich herausgestellt, dass diese Theorie Häufigkeiten und Mittelwerte teilweise erschreckend genau liefern kann. Wenn die Hälfte der Dynamik über Bord geworfen werden soll, dann müssen dennoch die statistischen Aussagen der Theorie im experimentell gesteckten Rahmen erhalten bleiben. Ob dies überhaupt möglich ist, soll hier untersucht werden. 

Die unstetige Dynamik hat die Eigenart, dass das Geschehen zum Erliegen kommt, wenn man immer nach derselben Information fragt. In traditioneller technischer Ausdrucksweise führt die Anwendung eines projektiven Messoperators zum "Kollaps" des Zustandsvektors auf einen Eigenraum des Messoperators. Danach führt die Wiederholung der projektiven Messung zu keinem Verlassen des Eigenraums mehr. Um ein Geschehen allein mit der unstetigen Dynamik am Laufen zu halten benötigen wird deswegen wengistens 2 unverträgliche Messoperatoren.

Während in gewöhnlicher Quantentheorie mit der Beobachtung üblicherweise Schluss ist, denn dann können die subjektiven Empfindungen gegen die Zahlen des mathematischen Modells verglichen werden, fordern wir ständige Beobachtungen, damit ein Geschehen überhaupt stattfinden kann. Da das menschliche Subjekt aber nur mit einer Seite der projektiven Messungen in Verbindung steht, sind ihm die anderen verborgen. Natürlich wäre es ontologisch das Einfachste anzunehmen, dass die anderen Messungen genauso wie die einen stattfinden, dass sie am Ende mit Empfindungen in Zusammenhang stehen, nur eben für andere Subjekte. 

\section{Grundlagen}

\subsection{Koordinatentransformationen, Produkträume und Verschränkung}

Zu tun: 

Indextransformationen, bei denen sich die Zahl der Indexe nicht ändert, entsprechen im Kontinuum Koordinatentransformationen. 

Bei Bildung von Produkträumen lassen sich die Indexe der Einzelräume auf einen einzigen Produktraumindex bijektiv abbilden. Wichtig: dies gilt auch für das Kontinuum! Auch dort existieren immer bijektive Abbildungen aufgrund der Gleichmächtigkeit der beteiligten Mengen.

Transformationen, die über Teilräume hinüberreichen, ändern i.A. deren Verschränkung. Solche, die sich nur auf Teilräume beschränken, erhalten die von-Neumann-Entropie.

\subsection{Orts-, Impuls-, Zeit- und Energieoperatoren}

Weil wir ihnen später begegnen werden, rufen wir uns die Matrixelemente spezieller Operatoren und die Komponenten ihrer Eigenvektoren in speziellen Vektorbasen in Erinnerung. Wir unterscheiden nicht zwischen abzählbaren und kontinuierlichen Dirac-Vektoren und verwenden durchgängig die Indexschreibweise, d.h. $f_{x_j} \equiv f(x)$.

Die Matrixelemente des Ortsoperator $\hat{x}$ und die Komponenten seiner Eigenvektoren $\ket{\psi_{x_j}}$ zu den Eigenwerten $x_j$ sind in der Ortsbasis $\{\,\ket{\psi_{x_j}}\,\}$
\begin{equation}
\bra{\psi_{x_j}}\ket{\,\hat{x}\,\psi_{x_k}}\ =\ x_{x_j x_k}\ =\ \delta_{x_j x_k}\,x_j
\quad\quad 
\bra{\psi_{x_k}}\ket{\psi_{x_j}}\ =\ \psi_{x_j x_k}\ =\ \delta_{x_j x_k}
\end{equation}

und in der (Orts-)Impulsbasis $\{\,\ket{\psi_{p_j}}\,\}$
\begin{equation}
\bra{\psi_{p_j}}\ket{\,\hat{x}\,\psi_{p_k}}\ =\ x_{p_j p_k}\ =\ 
\mathrm{i}\hbar\,\delta_{p_j p_k}\,\frac{\partial}{\partial p_k}
\quad\quad 
\bra{\psi_{p_k}}\ket{\psi_{x_j}}\ =\ \psi_{x_j p_k}\ =\ 
\frac{1}{\sqrt{2\pi}} e^{-\frac{i}{\hbar}x_j p_k}
\end{equation}

Analog sind die Matrixelemente des Zeitoperators $\hat{t}$ und die Komponenten seiner Eigenvektoren $\ket{\psi_{t_j}}$ zu den Eigenwerten $t_j$ in der Zeitbasis $\{\,\ket{\psi_{t_j}}\,\}$
\begin{equation}
\bra{\psi_{t_j}}\ket{\,\hat{t}\,\psi_{t_k}}\ =\ t_{t_j t_k}\ =\ \delta_{t_j t_k}\,t_j
\quad\quad 
\bra{\psi_{t_k}}\ket{\psi_{t_j}}\ =\ \psi_{t_j t_k}\ =\ \delta_{t_j t_k}
\end{equation}

und in der (Zeit-)Impuls- oder Energiebasis $\{\,\ket{\psi_{E_j}}\,\}$
\begin{equation}
\bra{\psi_{E_j}}\ket{\,\hat{t}\,\psi_{E_k}}\ =\ t_{E_j E_k}\ =\ 
-\mathrm{i}\hbar\,\delta_{E_j E_k}\,\frac{\partial}{\partial E_k}
\quad\quad 
\bra{\psi_{E_k}}\ket{\psi_{t_j}}\ =\ \psi_{t_j E_k}\ =\ 
\frac{1}{\sqrt{2\pi}} e^{\frac{i}{\hbar}t_j E_k}
\end{equation}

\subsection{Die Bornsche Regel}
%https://en.wikipedia.org/wiki/Born_rule

\begin{equation} 
p^{(0)\rightarrow(1)} =\ \Bigl| \bra{\psi^{(1)}}\ket{\psi^{(0)}} \Bigr| ^2
\end{equation}
Zu tun: Erläutern wie man sich den Prozess ursprünglich vorgestellt hat und wie man ihn sich mit Dekohärenztheorie vorstellt.


\section{Die Zeit in der Quantenmechanik}

\begin{equation} 
\hat{O}\ket{\psi} = 0 
\end{equation}

$\hat{O}$ soll in einem Produktraum $\mathscr{H} \otimes \mathscr{E}$ wirken. Hat die Gestalt 

\begin{equation} 
\hat{O} = \hat{H} \otimes \hat{1}^{\mathscr{E}} - \hat{1}^{\mathscr{H}} \otimes \hat{E}
\end{equation}

Produktansatz für $\ket{\psi}$

\begin{equation} 
\ket{\psi} = \ket{\psi}^{\mathscr{H}} \otimes \ket{\psi}^{\mathscr{E}}
\end{equation}

\begin{equation} 
\hat{H}\ket{\psi}^{\mathscr{H}} \otimes \ket{\psi}^{\mathscr{E}} 
\ - \ 
\ket{\psi}^{\mathscr{H}} \otimes \hat{E}\ket{\psi}^{\mathscr{E}} =\ 0
\end{equation}

Funktioniert für Schrödinger- und Dirac-Gleichung. Bsp. liefern...


Gleichungssystem nur noch gekoppelt über die Eigenwerte $E$
\begin{equation}
\begin{split}
\hat{H}\ket{\psi}^{\mathscr{H}} = E\ket{\psi}^{\mathscr{H}}\\
\hat{E}\ket{\psi}^{\mathscr{E}} = E\ket{\psi}^{\mathscr{E}}\ \ 
\end{split}
\end{equation}

Aufgrund der Linearität sind alle Linearkombinantionen von Produkten zum selben Eigenwert $E$ ebenfalls Lösungen.
\begin{equation}
\ket{\psi} = \sum_j\ \psi_j \ket{\psi_{E_j}}^{\mathscr{H}} \otimes\ \ket{\psi_{E_j}}^{\mathscr{E}}
\end{equation}


Damit Bornsche Regel zwischen solchen Lösungen

\begin{equation} 
p^{(0)\rightarrow(1)} =\ \sum_{jk} \psi_j^{(1)*}\psi_k^{(1)}\psi_j^{(0)}\psi_k^{(0)*}
\end{equation}

Damit von der Betrachtung ausgeschlossen alle, die $j \neq k$ Bestandteile haben.

\begin{equation}
\ket{\psi} = \sum_{jk}\ \psi_{jk} \ket{\psi_{E_j}}^{\mathscr{H}} \otimes\ \ket{\psi_{E_k}}^{\mathscr{E}}
\end{equation}

Bornsche Regel allg.

\begin{equation} 
p^{(0)\rightarrow(1)} =\ \sum_{jklm} \psi_{jk}^{(1)*}\psi_{lm}^{(1)}\psi_{jk}^{(0)}\psi_{lm}^{(0)*}
\end{equation}


---------------------------

\begin{equation} 
\label{eq:conditional_probability}
p^{(0)\rightarrow(1)} =\ \Bigl| \bra{\psi^{(1)}}\ket{\hat{U}(t_1-t_0)\,\psi^{(0)}} \Bigr|^2
\end{equation}

In der Energiebasis ist $\hat{U}$ bei einem nicht explizit zeitabhängigen Hamilton-Operator diagonal.

Matrixelemente in der Energiebasis
\begin{equation} 
U_{E_j E_k} = \delta_{E_j E_k} e^{-\frac{\mathrm{i}}{\hbar}E_j(t_1-t_0)}
= 
\delta_{E_j E_k}(e^{\frac{\mathrm{i}}{\hbar}E_j t_1})^* e^{\frac{\mathrm{i}}{\hbar}E_j t_0}
\end{equation}
In Komponenten
\begin{equation} 
p^{(0)\rightarrow(1)} =\ 
\Bigl| \sum_j
(\psi_{E_j}^{(1)\mathscr{H}} e^{\frac{\mathrm{i}}{\hbar}E_j t_1})^*
\psi_{E_j}^{(0)\mathscr{H}} e^{\frac{\mathrm{i}}{\hbar}E_j t_0}
\Bigr|^2
\ = \ 
\Bigl| \sum_j
(\psi_{E_j}^{(1)\mathscr{H}} \psi_{t_1 E_j}^{\mathscr{E}})^*
\psi_{E_j}^{(0)\mathscr{H}} \psi_{t_0 E_j}^{\mathscr{E}}
\Bigr|^2
\end{equation}

Das sind Born'sche Übergangswahrscheinlichkeiten für Vektoren aus dem $\mathrm{H}\otimes\mathrm{E}$ Raum. 

\begin{equation}
\ket{\psi} = \sum_j \psi_{E_j} \ket{\psi_{E_j}} \otimes  \psi_{t_0 E_j} \ket{\psi_{E_j}}
\end{equation}

\begin{equation}
\ket{\psi^\mathscr{H}} = \sum_j \psi_{E_j} \ket{\psi_{E_j}}
\quad\quad
\ket{\psi^\mathscr{E}} = \sum_j \psi_{E_j} \ket{\psi_{E_j}}
\end{equation}

\begin{equation}
\ket{\psi^{\mathscr{H}\otimes\mathscr{E}}} = \sum_{jk} \psi_{jk} \ket{\psi_{E_j}^\mathscr{H}} \otimes
\ket{\psi_{E_k}^{\mathscr{E}}}
\end{equation}
QM handelt also nur von Wahrscheinlichkeiten für einen Unterraum.
Kann nicht richtig sein, da wegen spez. Relat. sich das Weltgeschehen für einen relativ bewegeten Beobachter in einem anderen Unterraum von $\mathrm{H}\otimes\mathrm{E}$ abspielt.

Sollen die QM beider Beobachter gültige Theorien sein, dann muss sich für einen Beobachter seine QM in seinem Unterraum abspielen, während sich für den anderen Beobachter seine QM in einem anderen Unterraum abspielt. 

Sollen beide Beobachter dasselbe Weltgeschehen beobachten, dann muss es einen Filter geben, so dass jeder Beobachter nur seinen Unterraum des beobachten kann. Der Beobachter ist dieser Filter.

\section{Blockuniversen ersetzen absolute Zeit}
Sei $\ket{\psi}$ der Zustandsvektor einer Quantenwelt. Dann berechnet sich in der nichtrelativistischen Quantenmechanik die bedingte Wahrscheinlichkeit dafür, die Quantenwelt zur Zeit $t_1$ im Zustand $\ket{\psi^{(1)}}$ anzutreffen unter der Bedingung, dass sie zur Zeit $t_0$ im Zustand  $\ket{\psi{(0)}}$ war, zu
\begin{equation} 
\label{eq:conditional_probability}
p^{(0)\rightarrow(1)} =\ \Bigl| \bra{\psi^{(1)}}\ket{\hat{U}(t_1-t_0)\,\psi^{(0)}} \Bigr|^2
\end{equation}
Dabei ist $\hat{U}$ der unitäre Zeitentwicklungsoperator, der parametrisch von einem als absolut angesehenen Zeitparameter $t$ abhängt. Wir beschränken uns hier auf nicht explizit zeitabhängige Welten, d.h. $\hat{U}$ kann mit einem hermiteschen Operator $\hat{H}$ in der Form
\begin{equation}
\label{eq:time_evolution}
\hat{U}(t_1,t_0)=\exp \left(-\tfrac {\mathrm {i} }{\hbar}\hat{H}\cdot (t_1-t_0)\right)
\end{equation}
geschrieben werden. 


-----------------
diskret

Schrödi + 1 Kollaps
\begin{equation} 
p^{(0)\rightarrow(1)} =\ \Bigl| \sum_{x'x} \psi^{(1)*}_{x'}\ U(\Delta t)_{x'x}\ \psi^{(0)}_x \Bigr|^2
\end{equation}
Elementares $\Delta t$ Planck-Zeit $\hat{U} = \hat{u}^{t/t_{Planck}}$
\begin{equation} 
p^{(0)\rightarrow(1)} =\ \Bigl| \sum_{x'x} \psi^{(1)*}_{x'}\ u_{x'x}\ \psi^{(0)}_x \Bigr|^2
\end{equation}
1 Zwischenbasis, 2 Kollapse
\begin{equation} 
\ket{\psi}^{(0)} =  \sum_{x} \psi^{(0)}_x \ket{x} \otimes \ket{t_0} \quad\quad
\ket{\psi}^{(1)} =  \sum_{x} \psi^{(1)}_x \ket{x} \otimes \ket{t_1}
\end{equation}
Zwischenbasis $\{\ket{z}\}$ $N \cdot M$ Elemente
\begin{equation} 
\begin{matrix}
\ket{x} \otimes \ket{t} = \sum_z V_{xt\, z}\ \ket{z} && 
\ket{z} = \sum_{xt} V^*_{z\, xt}\ \ket{x} \otimes \ket{t} 
\\ \\
\bra{x} \otimes \bra{t} = \sum_z V^*_{z\, xt}\ \bra{z} &&
\bra{z} = \sum_{xt} V_{xt\, z}\ \bra{x} \otimes \bra{t} 
\end{matrix}
\end{equation}
\begin{equation} 
p^{(0)\rightarrow(z)} =\ \Bigl| \bra{z} \hat{V} \left(\ket{x}\otimes\ket{t_0}\right)\Bigr|^2
= \Bigl|\sum_{xt\, x'} \psi^{(0)}_{x'} \bra{x'} \otimes \bra{t} V_{xt\, z}\ \ket{x} \otimes \ket{t_0}\Bigr|^2 
\end{equation}
\begin{equation} 
p^{(0)\rightarrow(z)} =\ \Bigl|\sum_{x} \psi^{(0)}_{x} V_{xt_0\, z}\ \Bigr|^2 
\end{equation}
analog
\begin{equation} 
p^{(z)\rightarrow(1)} =\ \Bigl|\sum_{x} \psi^{(1)}_{x} V_{xt_1\, z}\ \Bigr|^2 
\end{equation}
Gesamtübergangswahrscheinlichkeit
\begin{equation} 
p^{(0)\rightarrow(1)} =\ \sum_z\ p^{(0)\rightarrow(z)}\ p^{(z)\rightarrow(1)}
= \ \sum_{z}\ \Bigl(\ \Bigl|\sum_{x} \psi^{(1)}_{x} V_{xt_1\, z}\ \Bigr|^2\ \Bigl|\sum_{x} \psi^{(0)}_{x} V_{xt_0\, z}\ \Bigr|^2\ \Bigr)
\end{equation}
\begin{equation} 
p^{(0)\rightarrow(1)} 
= \ \sum_{z}\ \Bigl|\sum_{x'x} \psi^{(1)}_{x'} V_{x't_1\, z}\ V_{xt_0\, z}\ \psi^{(0)}_{x} \Bigr|^2  
\end{equation}

%\begin{equation} 
%p^{(0)\rightarrow(1)} 
%= \ \sum_{z}\ \Bigl|\sum_{x'x} \psi^{(1)}_{x'} v_{z\, x'}\ w_{z\, x}\ \psi^{(0)}_{x} %\Bigr|^2  
%\end{equation}
Also 
\begin{equation} 
\boxed{
\quad\Bigl| \sum_{x'x} \psi^{(1)*}_{x'}\ u_{x'x}\ \psi^{(0)}_x \Bigr|^2 
\ =\ \sum_{z}\ \Bigl|\sum_{x'x} \psi^{(1)}_{x'} V_{x't_1\, z}\, V_{xt_0\, z}\ \psi^{(0)}_{x} \Bigr|^2 \quad
}
\end{equation}

Ausmultipliziert links
\begin{equation} 
\sum_{x'''x''x'x}\psi^{(1)}_{x'''}\psi^{(0)*}_{x''}\psi^{(1)*}_{x'}\psi^{(0)}_x
\ u^*_{x'''x''}\ u_{x'x}
\end{equation}
rechts
\begin{equation} 
\sum_{z\,x'''x''x'x}\ 
\psi^{(1)}_{x'''}\psi^{(0)*}_{x''}\psi^{(1)*}_{x'}\psi^{(0)}_{x}
\, V_{x'''t_1\, z}\, V^*_{x''t_0\, z}\, V^*_{x't_1\, z}\, V_{xt_0\, z}
\end{equation}
Muss für alle $\ket{\psi^{(0)}}$ und $\ket{\psi^{(1)}}$ gelten, deshalb
\begin{equation} 
\boxed{
u^*_{x'''x''}\ u_{x'x} = \sum_z \, V_{x'''t_1\, z}\, V^*_{x''t_0\, z}\, V^*_{x't_1\, z}\, V_{xt_0\, z} 
}
\end{equation}
Unitarität von $u$ und $V$ sind zusätzlich zu fordern!
\begin{equation} 
\sum_{x''} u_{x'x''}\, u^*_{xx''} = \delta_{x'x} \quad\quad 
\sum_{z} V_{x't'\, z} V^*_{xt\, z} = \delta_{x'x}\ \delta_{t't} \quad\quad
\sum_{xt} V_{xt\, z'} V^*_{xt\, z} = \delta_{z'z}
\end{equation}

-----------------

(\ref{eq:time_evolution}) diagnonalisiert durch unitäre Matrix $W$ im x-Unterraum. D.h. auf $V$ wirkt $W \otimes \mathrm{1}$.
\begin{equation}
U_{x'x\,t't} = e^\mathrm{i\varphi_x(t'-t)}\,\delta_{x'x}
\end{equation}
\begin{equation}
\begin{split}
\boxed{
e^\mathrm{i(\varphi_x-\varphi_{x''})(t'-t)}\,\delta_{x'''x''}\,\delta_{x'x}\ =\ \sum_z \, V_{x'''t'\, z}\, V^*_{x''t\, z}\, V^*_{x't'\, z}\, V_{xt\, z} 
} \\
\varphi_x \in \mathbb{R} \\ V_{xt\, z} \in \mathbb{C} \\
x \in \{1,2,...,N_x\}\\ t \in \{1,2,...,N_t\}\\ z \in \{1,2,...,N_x \cdot N_t\} 
\end{split}
\end{equation}

Nur noch Unitarität von $V$ ist zusätzlich zu fordern!
\begin{equation} 
\sum_{z} V_{x't'\, z} V^*_{xt\, z} = \delta_{x'x}\ \delta_{t't} \quad\quad
\sum_{xt} V_{xt\, z'} V^*_{xt\, z} = \delta_{z'z}
\end{equation}


-----------------
ab hier überarbeiten

Wenn die Quantenzahlen kontinuierlich sind, dann drückt sich (\ref{eq:conditional_probability}) in den Komponenten zu einer weiteren Basis $f$ so aus\footnote{Es ist $x(f) \equiv \bra{f}\ket{x}$ und $U(t,f,f') \equiv \bra{f}\ket{\hat{U}(t)f'}$.}
\begin{equation} 
\label{eq:conditional_probability_components}
p(x_1,t_1) \Biggl|_{x_0,t_0}
=\ \Bigl| \iint \mathrm{d}f\mathrm{d}f'\, x_1^*(f)\, U(t_1-t_0,f,f')\, x_0(f')\, \Bigr|^2
\end{equation}
Wenn wir von der parametrischen absoluten Zeit wegkommen wollen, dann wünschen wir uns so etwas wie
\begin{equation} 
\label{eq:conditional_probability_relat}
p(x_1,t_1)\Biggl|_{x_0,t_0} =\ \Bigl| \bra{x_1,t_1}\ket{x_0,t_0} \Bigr|^2
\end{equation}
womit die bedingte Wahrscheinlichkeit gemeint ist, die Quantenwelt in den Zustand $\ket{x_1,t_1}$ zu bringen unter der Vorraussetzung, dass sie im Zustand $\ket{x_0,t_0}$ ist.  $x$ und $t$ nummerieren nun gemeinsam die Zustandsvektoren so wie in relativistischen Quantenfeldtheorien. Jeder Zustandsvektor stellt ein bestimmtes \href{https://de.wikipedia.org/wiki/Blockuniversum}{Blockuniversum} dar.

Jeder unstetige Übergang von einem Zustandsvektor zum nächsten wird in der Physik traditionell als \emph{Messung} bezeichnet. 

\section{Die Rolle des Subjekts}

Subjektiv erleben wir Ereignisse in der (psychischen) Zeit. $\ket{x,t}$ steht nun aber nicht für ein Ereignis der Erfahrung eines Zustands $\ket{x}$ in der Zeit $t$, sondern im Allgemeinen für einen x,t-verschränkten Zustand\footnote{Irgendeine Funktion zweier Veränderlicher $f(x,t)\equiv\bra{x,t}\ket{f}$ zerfällt im Allgemeinen nicht in ein Produkt $g(x)h(t)$ und ist damit im Allgemeinen x,t-verschränkt.}. Da das Subjekt keine Zustände erfährt, die in der Zeit verschränkt sind, muss es an der Messung in einer Weise beteiligt sein, so dass auf entschränkte Produktzustände projiziert wird. Das Subjekt muss also dafür sorgen, dass auf Produktzustände $\ket{x_i}\otimes\ket{t_i}$ projiziert wird, wobei $\ket{t_i}$ eine Zeiteigenfunktion\footnote{in Komponenten also eine Delta-Distribution $\delta(t-t_i)$} zum Zeiteigenwert $t_i$ ist. Dadurch erfährt das Subjekt die physikalische Zeit $t_i$ zusammen mit dem neuen Weltzustand $\ket{x_i}$. Wiederholt sich dies, dann erfährt das Subjekt eine Folge von Weltzuständen $\ket{x_i}$, die durch die \href{https://de.wikipedia.org/wiki/Zeitdilatation}{Eigenzeiten} $t_i$ nummeriert sind, also bei genügender Feinheit des Prozesses vermeintlich eine quasikontinuierliche Abfolge $\ket{x(t)}$.

Sei $\ket{y_0}$ ein Blockuniversum vor der Messung, oder in Komponenten zur Produktbasis $f,t$ ausgedrückt $y_0(f,t)$. Dann ist die Übergangswahrscheinlichkeit (\ref{eq:conditional_probability_relat}) in einen subjektiv erfahrbaren Zustand $\ket{y_1} = \ket{x_1}\otimes\ket{t_1}$
\begin{equation} 
\label{eq:conditional_probability_subject}
p(x_1,t_1)\Biggl|_{y_0}\ =\ \Bigl| \iint \mathrm{d}f\mathrm{d}t\  x_1^*(f)\delta(t-t_1)\, y_0(f,t)\, \Bigr|^2
\ =\ \Bigl| \int \mathrm{d}f\, x_1^*(f)\, y_0(f,t_1)\, \Bigr|^2
\end{equation}
Vergleichen wir dies nun mit (\ref{eq:conditional_probability_components}), dann lesen wir ab
\begin{equation} 
y_0(f,t_1)\ =\ \int \mathrm{d}f'\, U(t_1-t_0,f,f')\, x_0(f')
\end{equation}


\end{document}
