% -------------------- Einstellungen Seitenaufbau ------------------------------
% Seiteneinstellungen laden
\usepackage[nouppercase]{scrpage2}
% Keine Kopf/Fußzeile auf leeren Seiten
\usepackage{emptypage}
% Zeilenabstände einstellen auf standardmäßig 1,5fachen Zeilenabstand
\usepackage[singlespacing]{setspace}
% Seitenränder einstellen
\setlength{\topskip}{10.5pt} % Verhindern einer Fehlermeldung

% Seitengröße auf a4paper oder a5paper einstellbar (1. oder 2. usepackage wählen)
\usepackage[a5paper,headheight=1.5\baselineskip,top=25mm,lines=38,heightrounded=true,bindingoffset=12mm,textwidth=109mm,footnotesep=7mm]{geometry}
%\usepackage[a4paper,headheight=1.5\baselineskip,top=25mm,lines=46,heightrounded=true,bindingoffset=15mm,textwidth=160mm]{geometry}
% Zeilen auf der Seite verteilen ( Es wird kein Ausgleich des unteren Seitenrandes durch Dehnung der Absatzabstände durchgeführt.) 
\raggedbottom     
% Abstand zwischen Textkörper und Linie in der Kopfzeile
\setlength{\headsep}{8mm}
% Abstand zwischen Textkörper und Unterkante Fußzeile (Seitenzahlen)
\setlength{\footskip}{10mm}
% Verhindert  Erstzeileneinzüge
\setlength{\parindent}{0pt}
\setlength{\parskip}{2.5mm}
% Schusterjungen (einzelne Zeile unten auf der Seite) unterdrücken
\clubpenalty = 10000 
% Hurenkinder (einzelne Zeile oben auf der Seite) unterdrücken
\widowpenalty = 10000
\displaywidowpenalty = 10000
% Linie in der Kopfzeile definieren
\setheadsepline{0.5pt}
% Position und Größe der Fußnoten
\deffootnote[1.0em]{1.0em}{0em}{\textsuperscript{\thefootnotemark\ }}
% Mindestfüllgrad einer Seite mit einem Gleitobjekt
\renewcommand{\floatpagefraction}{0.7}
% Maximale Größe eines Gleitobjekts am unteren Seitenrand
\renewcommand{\topfraction}{0.8}
% Maximale Größe eines Gleitobjekts am oberen Seitenrand
\renewcommand{\bottomfraction}{0.8}
% mögliche Abstandsvergrößerung innerhalb einer Zeile bei unschönem Zeilenumbruch
\setlength{\emergencystretch}{4pt}
% Mindestanteil an Text auf einer Seite mit Gleitobjekt
\renewcommand{\textfraction}{0.1}
% Seitenbegrenzungen anzeigen, zum Überprüfen der Seitendarstellung
%\usepackage{showframe}

% -------------------- Einstellungen Zeichensatz ------------------------------
% Erscheinungsbild auf Deutsch umstellen
\usepackage{babel}
% Umlaute zulassen
\usepackage[T1]{fontenc}
% ß als Zeichen zulassen
\usepackage[latin1]{inputenc}
% Inhaltsverzeichnis richtig darstellen
\usepackage[titles]{tocloft}
% Auch bei den Kapiteln Punkte darstellen
\renewcommand{\cftchapdotsep}{\cftdotsep}
\renewcommand{\cftchapleader}{\cftdotfill{\cftchapdotsep}}
% Seitenzahlen bei Kapitel in Serifenloser Schriftart darstellen
\renewcommand{\cftchappagefont}{\fontfamily{phv}\normalsize\bfseries}
% Verzeichnisse aktualisieren

%Fonts im Inhaltsvz.
\renewcommand\cftchapfont{\fontfamily{phv}\normalsize\bfseries}
\renewcommand\cftsecfont{\fontfamily{phv}\fontsize{11}{11}}
%Fonts in Kapiteln und sections...
%\renewcommand\cftchappagefont{\fontfamily{phv}\normalsize\bfseries}
\renewcommand\cftsecpagefont{\fontfamily{phv}\fontsize{11}{11}}

\usepackage{makeidx}
% Adobe Times als Standardschriftart einstellen
\usepackage{newtxtext}
\usepackage{newtxmath}
% Schriftarten und -größen für die Überschriften vorgeben
%\addtokomafont{chapter}{\fontfamily{phv}\fontsize{17}{17}\bfseries}

%\addtokomafont{chapter}{\fontfamily{phv}\fontsize{17}{17}\bfseries}
%\addtokomafont{section}{\fontfamily{phv}\fontsize{13}{13}\bfseries}
%\addtokomafont{subsection}{\fontfamily{phv}\fontsize{11}{11}\bfseries}
%\addtokomafont{subsubsection}{\fontfamily{phv}\fontsize{9}{9}\bfseries}

\addtokomafont{chapter}{\fontfamily{phv}\fontsize{17}{19}\bfseries}
\addtokomafont{section}{\fontfamily{phv}\fontsize{13}{16}\bfseries}
\addtokomafont{subsection}{\fontfamily{phv}\fontsize{11}{14}\bfseries}
\addtokomafont{subsubsection}{\fontfamily{phv}\fontsize{9}{12}\bfseries}

% Größen der Beschriftungen vorgeben
\addtokomafont{caption}{\footnotesize}
\setkomafont{captionlabel}{\footnotesize}
% Größe der Kopf- und Fußzeile vorgeben
\setkomafont{pageheadfoot}{\footnotesize} 
% Größe der Seitenzahl
\setkomafont{pagenumber}{\normalsize}

% Farben im Dokument zulassen
\usepackage{color}
% Textfarbe schwarz definieren
\color[cmyk]{0,0,0,1}
% Bezeichnung für Überschriften 1cm vom linken Rand beginnen
\renewcommand*{\chapterformat}{\makebox[1.1cm][l]{\thechapter\autodot}}
\renewcommand*{\sectionformat}{\makebox[1.1cm][l]{\thesection\autodot}}
\renewcommand*{\subsectionformat}{\makebox[1.1cm][l]{\thesubsection\autodot}}
% Bezeichnungsnamen angeben
\addto\captionsngerman{\renewcommand{\figurename}{Abbildung}}
\addto\captionsngerman{\renewcommand{\tablename}{Tabelle}}

% Ansicht Literaturverzeichnis
%\bibliographystyle{plaindin}

% -------------------- Einstellungen weiterer Pakete ------------------------------
% Einbinden von Bildern ermöglichen
\usepackage{graphicx}	
% Gedrehte Objekte ermöglichen
\usepackage{rotating}
% Erweiterte Tabellenumgebung
\usepackage{tabularx}
% Erweiterte Flatersatz-Kommandos
\usepackage{ragged2e}
% Linksbündiger Flattersatz in den Bezeichnungen
%\usepackage[justification=RaggedRight]{caption}
\usepackage[justification=justified]{caption}
\captionsetup[subfigure]{justification=RaggedRight}
% Neuer Spaltentyp "L" mit Breitenangabe für linksbündigen Flattersatz
\newcolumntype{L}[1]{>{\RaggedRight\arraybackslash}p{#1}}
% Mathematische Symbole
\usepackage{amsmath,amssymb}
% Zeilen in Tabellen können verbunden werden
\usepackage{multirow}
% Zusätzliche Textsymbole zur Verfügung stellen
\usepackage{textcomp}
% Operatorensymbole definieren
\newcommand{\real}{\operatorname{Re}}				% Realteil
\newcommand{\opdiv}{\operatorname{div}}			% Divergenzoperator
\newcommand{\rot}{\operatorname{rot}}				% Rotationsoperator
\newcommand{\grad}{\operatorname{grad}}			% Gradientenoperator
\newcommand{\imag}{\operatorname{Im}}				% Imaginärteil
\newcommand{\imein}{\operatorname{j}}				% imaginäre Einheit j
% Darstellung urls mit Zeilenumbruch
\usepackage[hyphens]{url}
% Erweiterte Listenanweisungen
\usepackage{etoolbox}
% Zeilenumbrüche in urls nach folgenden Zeichen
\appto\UrlBreaks{\do\a\do\b\do\c\do\d\do\e\do\f\do\g\do\h\do\i\do\j\do\k\do\l\do\m\do\n\do\o\do\p\do\q\do\r\do\s\do\t\do\u\do\v\do\w\do\x\do\y\do\z\do\/\do\.}
% Einzug im Abbildungsverzeichnis zu Null setzen
\renewcommand{\cftfigindent}{0cm}
% Einzug im Tabellenverzeichnis zu Null setzen
\renewcommand{\cfttabindent}{0cm}
% Darstellung und Verlinkungen im pdf-Dokument einstellen
\usepackage[hidelinks,								% Links als normaler Text darstellen
	pdfpagemode = UseNone,							% Lesezeichen im pdf-Reader nicht anzeigen
	pdfpagelayout = TwoColumnRight,			% Seitenanzeige des pdf-Dokuments angeben
	pdfauthor = {\autor},								% Autor des pdf-Dokuments
	pdftitle = {\pdftitle}]							% Titel des pdf-Dokuments
	{hyperref}
% deutsches Abkürzungsverzeichnis erstellen
\usepackage[german]{nomencl}
% Befehl für einen Eintrag im Abkürzungsverzeichnis in "\sym" umbennen
\let\sym\nomenclature
% Name des Abkürzungsverzeichnis ändern
\renewcommand{\nomname}{Abkürzungs- und Symbolverzeichnis}
% Spaltenbreite der Formelzeichen auf "20 %" der Textbreite setzen
\setlength{\nomlabelwidth}{.2\textwidth}
% Einheiten in die Bezeichnung mit aufnehmen und rechtsbündig setzen
\newcommand{\nomunit}[1]{\renewcommand{\nomentryend}{\hspace*{\fill}#1}}
% Zeilenabstände verkleineren auf normalen Textabstand
\setlength\nomitemsep{-\parsep}
% Abkürzungsverzeichnis erzeugen
\makenomenclature
% Weitere Verzeichnisse erzeugen
\makeindex
% Blindtext zur Textdarstellung in der Vorlage ermöglichen
\usepackage{blindtext}


\AtBeginDocument{% 
  \newcaptionname{ngerman}\equationname{Formel}% 
  \newcaptionname{ngerman}\listequationname{Formelverzeichnis}% 
}

\DeclareNewTOC[ 
  indent=0pt,
  hang=2em,
  type=equation,
  name={Gl.}, 
  types=equations, 
  listname={Formelverzeichnis}, 
]{equ} 
\newcommand{\equationentry}[2][\theequation]{% 
  \addxcontentsline{equ}{equation}[{#1}]{\kern 1em #2}% 
} 
\BeforeStartingTOC[equ]{\def\autodot{:}} 