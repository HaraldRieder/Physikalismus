\documentclass[12pt]{article}
\usepackage[margin=2cm, bindingoffset=0cm]{geometry}
%\usepackage[german]{babel} 
\usepackage[ngerman]{babel} %sudo apt-get install texlive-lang-german
\usepackage{hyperref} % web links etc.
\usepackage[parfill]{parskip}
\usepackage[utf8]{inputenc}
\usepackage[dvipsnames]{xcolor}
\usepackage{tcolorbox}
\usepackage{helvet} 
\usepackage{framed}
\usepackage{anyfontsize}
\usepackage{csquotes}
\usepackage{mathrsfs}
\usepackage{physics}
\setcounter{MaxMatrixCols}{16}
\usepackage{amssymb}
\usepackage{MnSymbol}
\MakeOuterQuote{"}
\definecolor{quotecolor}{rgb}{0.8,0.9,1}
\renewcommand{\familydefault}{\sfdefault} 
\setlength\parindent{0pt}
\tcbset{boxrule=0pt,colback=quotecolor,arc=5pt,auto outer arc,left=5pt,right=5pt,boxsep=5pt}
%  width=0.9\textwidth,

\setcounter{secnumdepth}{-1}
\begin{document}
\title{\fontsize{25}{25}\selectfont \textbf{Über psychische und physikalische Zeit}}
\author{Harald Rieder}
\date{\today}
\maketitle

%\begin{abstract}

%\end{abstract}

\tableofcontents

\section{Motivation}

Der Erfolg der relativistischen Physik lehrt uns, dass die Zeit irgendwie gleichrangig neben dem Ort zu stehen hat. Wechselt ein Beobachter seine Perspektive auf bestimmte Art, sehen  Koordinaten, die zuvor wie reine Ortskoordinaten ausgesehen haben, plötzlich ein wenig wie Zeitkoordinaten aus, und umgekehrt. Ein solcher Perspektiv-Wechsel heißt in der speziellen Relativitätstheorie \emph{Lorentz-Transformation}.\footnote{Ein Beispiel aus der allgemeinen Relativitätstheorie: Beim Passieren des Ereignishorizontes der Schwarzschild-Metrik tauschen Radialterm und Zeitterm ihre Vorzeichen aus. Nach dem starken Äquivalenzprinzip ist so ein Standortwechsel lokal äquivalent zu einem Wechsel der Perspektive.} 

Der Alltag lehrt uns, dass die Zeit kontinuierlich vergeht, und zwar in eine bestimmte Richtung. Denn das Alltagsgeschehen scheint sich meistens nicht umkehren zu lassen. Eine zerschellte Tasse setzt sich nicht mit der Zeit wieder zu einer heilen Tasse zusammen und kehrt unter Abkühlung auf den Tisch zurück. Solches Alltagsgeschehen wird durch den 2. Hauptsatz der Thermodynamik erfolgreich modelliert.

Der Erfolg der Physik insgesamt lehrt uns, dass Ortsraum keine eingebaute Richtung hat. Es gibt keinen Satz der behauptet, dass wenn man in eine bestimmte Richtung geht, die Tassen immer kaputter werden müssen.

Somit haben wir 3 erfolgreich anwendbare Konzepte, die im Widerspruch zu stehen scheinen. Raum und Zeit sind wie Bruder und Schwester. Die Zeit hat eine eingebaute Richtung, der Raum aber nicht. Und nun?

Darum schauen wir uns an, wozu uns ein paar unkonventionelle Annahmen führen können:

\begin{itemize}
\item In der Physik vergeht keine Zeit. Die Natur der physikalischen Zeit ist wie die Natur des physikalischen Ortsraums.
\item In der Psyche vergeht Zeit. Sie ist integraler Bestandteil des Bewusstwerdens, des Geistes. 
\item Es muss einen Mechanismus geben, der die psychische Zeit mit einem Bestandteil der physikalischen Welt so koppelt, dass es wenigstens für die menschliche Psyche den Anschein hat, die physikalische Zeit würde objektiv vergehen.
\end{itemize}

\section{Idee zu einer Quantenuhr}

In der Quantenmechanik spielt sich das Geschehen in einem vieldimensionalen Konfigurationsraum ab, dem Hilbert-Raum. Ein Ort existiert in diesem Raum zunächst nicht. Erst durch Festlegung auf eine Basis lassen sich komplexwertige Funktionen erstellen. Diese Funktionen stehen für unendlich viele Amplituden, die durch einen kontinuierlichen reellwertigen Index $x$ durchnummeriert werden.

Durch Beobachtung von außen, sowie durch Symmetrien, die wir aus der Erfahrung her einem dreidimensionalen Ortsraum zuschreiben, gelingt uns der Anschluss. Es sind unendlich viele Basen wählbar, doch nur bestimmte Wahlen führen zu einer Indexierung, bei der wir den Index als Ortskoordinate interpretieren können. Die Amplitude eines abstrakten Zustandsvektors $\ket{\psi}$ an einem bestimmten Ort liefert uns das Skalarprodukt mit dem abstrakten Vektor einer "Ortsbasis"
\begin{equation} 
\psi(x) \equiv \bra{x}\ket{\psi} 
\end{equation}
Wenn wir die physikalische Zeit ähnlich wie den Ort behandeln wollen, dann muss die Zeitkoordinate $t$ ebenso ein reellwertiger kontinuierlicher Index sein. Das heißt
\begin{equation} 
\label{eq:psi_xt}
\psi(x,t) \equiv \bra{x,t}\ket{\psi} 
\end{equation}
Die Indizes $x$ und $t$ nummerieren nun zusammen eine Produktbasis aus Orts- und Zeiteigenvektoren. Wir könnten aufgrund der Gleichmächtigkeit von $\mathbb{R}$ mit $\mathbb{R}^2$ diesen Index durch einen gemeinsamen reellwertigen Index $i = i(x,t)$ ersetzen und kämen damit wieder auf die Form
\begin{equation}
\psi(i) \equiv \bra{i}\ket{\psi} 
\end{equation}
wobei $\ket{\psi}$ wie in (\ref{eq:psi_xt}) der abstrakte Zustandsvektor im Produktraum wäre.

Wir nehmen an, dass der Beobachter aus seinem Hilbert-Raum $\mathscr{H}_X$ heraus nicht in der Lage dazu ist, Zeit "direkt" in Erfahrung zu bringen. Diese Annahme drückt sich in der Quantenmechanik so aus, dass in Matrixelementen $H(X,x)$ von Wechselwirkungs-Hamiltonians die Zeit nicht auftritt. Ein Beobachter muss eine Zeigerstellung, einen Ort, ablesen, um von dort auf die Zeit im Uhrenzustand zu schließen. 

\begin{figure}[!h]\begin{center}
  \includegraphics[width=6cm]{Quantenuhr.png}
  \caption{Indirektes Ablesen der Uhrzeit}
  \label{fig:sr_game}
\end{center}\end{figure}

Ein guter Uhrenzustand kann also ein Zustandsvektor im Produktraum $\mathscr{H}_x \otimes \mathscr{H}_t$ sein, der Orts- und Zeitunterräume maximal verschränkt. Wenn $\delta(x-\xi)$ die Amplituden von Ortseigenvektoren im Ortsunterraum in der Ortsdarstellung sind, und $\delta(t-\tau)$ die Amplituden von Zeiteigenvektoren im Zeitunterraum in der Zeitdarstellung, dann sind\footnote{Fortan lassen wir die Integralgrenzen weg, wenn sie im Unendlichen liegen.}
\begin{equation} 
\label{eq:psi_clock}
\psi(x,t) \sim \int_{-\infty}^{\infty} d\chi \delta(x-\chi) \delta(t-\chi) = \delta(x-t)
\end{equation}
die Amplituden eines maximal verschränkten Zustands, der aus der Beobachtung von $\xi$ sicher auf die Zeit $\tau$ schließen lässt. Durch die Beobachtung (oder Messung) "kollabiert die Überlagerung"
\begin{equation} 
\label{eq:collapse}
\int d\chi \delta(x-\chi) \delta(t-\chi) \quad \rightarrow \quad \delta(x-\chi)\delta(t-\chi)
\end{equation}
Dadurch ist die Uhr zunächst kaputtgegangen. Denn durch verträgliche Messungen, also wiederholte Ortsmessungen, werden wir immer wieder nur diesen Zustand und damit die Zeit $\chi$ antreffen. Es fehlt also ein Mechanismus, der die Uhr wieder scharfschaltet, sie in ihren verschränkten Zustand zurückbringt.

Im Artikel \emph{\href{https://docs.google.com/document/d/1OrmVETmnBSe5c0CpTbKH8Vq5pWFuB8QUez-KqHTaarQ/edit?usp=sharing}{Ideas about a Quantum Theory without Process Type 2}} wird so ein Mechanismus vorgestellt. Dort wird postuliert, dass jeder bewusste Beobachter einerseits eine Teilung des Hilbertraums vornimmt, andererseits die Änderung der Verschränkungen, wie sie erst durch die jeweilige Teilung aus den jeweils vorliegenden Zustandsvektoren entstehen, erlebt, ja sie womöglich willentlich herbeiführen kann, wodurch dann \emph{aus Sicht des jeweiligen Beobachters} die quantenmechanische Überlagerung kollabiert und am Ende ein reiner Produktzustand vorliegt. Mindestens 2 solcherart an den physikalischen Kanal angeschlossene Beobachter (\emph{conscious splits}) sind notwendig, um ein Geschehen am Laufen zu halten. Unsere Uhr soll von einem äußeren Beobachter wie gezeigt hin und wieder abgelesen werden. Wir benötigen also noch einen weiteren "internen" Beobachter, der den x,t-Produktraum auf andere Weise teilen muss als der externe Beobachter. Der interne Beobachter teilt den Produktraum dazu nicht in x- und t-Basen, sondern in eine Basis aus x,t-verschränkten Zuständen und eine Basis, die den ganzen Rest enthält.

Um zu sehen, wie es läuft, betrachten wir ein einfaches Beispiel...

\section{Eine Quantenuhr aus 2 Qutrits}

Unsere einfache Quantenuhr soll nur 3 diskrete Zeigerstellungen haben: $\ket{x=0}, \ket{x=1}, \ket{x=2}$. Sie soll auch nur 3 Zeitpunkte messen können: $\ket{t=0}, \ket{t=1}, \ket{t=2}$. Wir lassen später der Übersichtlichkeit halber x und t weg, x soll links stehen, t rechts. Das heißt zum Beispiel $\ket{00}$ soll für $\ket{x=0}\ket{t=0}$ stehen. 

Wir haben es mit einem Produktraum aus 2 Qutrits, 1 Raum- und 1 Zeit-Qutrit zu tun. Für eine Orthogonalbasis sind somit 9 Basisvektoren notwendig. Wir bilden diese aus den Produkten der 3 Raum- und 3 Zeit-Eigenvektoren.

Einen x,t-verschränkten Zustand können wir zum Beispiel so bilden
\begin{equation*}
\frac{1}{\sqrt{2}} \left(\, \ket{x=0}\otimes\ket{t=0} + \ket{x=1}\otimes\ket{t=1} \,\right) \equiv 
\frac{1}{\sqrt{2}} \left(\, \ket{00} + \ket{11} \,\right)
\quad\quad\hat{=}\quad\quad
\frac{1}{\sqrt{2}}
\begin{pmatrix}
1 \\ 0 \\ 0 \\ 0 \\ 1 \\ 0 \\ 0 \\ 0 \\ 0 
\end{pmatrix}
\quad
\begin{matrix}
00 \\ 01 \\ 02 \\ 10 \\ 11 \\ 12 \\ 20 \\ 21 \\ 22 
\end{matrix}
\end{equation*}
und einen unverschränkten Zustand so
\begin{equation*}
\frac{1}{\sqrt{2}}\, \ket{x=1}\otimes \left(\,\ket{t=0} + \ket{t=2} \,\right) \equiv 
\frac{1}{\sqrt{2}} \left(\, \ket{10} + \ket{12} \,\right)
\quad\quad\hat{=}\quad\quad
\frac{1}{\sqrt{2}}
\begin{pmatrix}
0 \\ 0 \\ 0 \\ 1 \\ 0 \\ 1 \\ 0 \\ 0 \\ 0 
\end{pmatrix}
\quad
\begin{matrix}
00 \\ 01 \\ 02 \\ 10 \\ 11 \\ 12 \\ 20 \\ 21 \\ 22 
\end{matrix}
\end{equation*}
Dies ist die Perspektive des äußeren Beobachters. Die Uhr soll ihm verschränkte Zustände anbieten, die aus Linearkombinationen von $\ket{\chi\chi}$ Vektoren zusammengesetzt sind. Er entscheidet sich dann für einen der Vektoren mit einer Wahrscheinlichkeit gemäß der Bornschen Regel. 

Daraufhin muss die Uhr wieder scharfgeschaltet werden, wozu wir den internen Beobachter brauchen. Für den internen Beobachter müssen alle $\ket{\chi\chi}_{ext}$ Zustände verschränkt aussehen, damit er tätig wird. Um auf seine Perspektive zu wechseln, brauchen wir eine unitäre Matrix $U$ im Produktraum, zum Beispiel\footnote{Elemente mit Wert 0 lassen wir der Übersichtlichkeit halber nun öfter weg.}
\begin{equation}
U\ =\ 
\begin{pmatrix}
\label{eq:U}
\pmb{a_0} &&&&& \pmb{a_2} &&&&& \pmb{a_1} \\
  & 1 &   &   &   &   &   &   &   &   &   \\
  &   & 1 &   &   &   &   &   &   &   &   \\
  &   &   & 1 &   &   &   &   &   &   &   \\
\pmb{a_1} &&&&& \pmb{a_0} &&&&& \pmb{a_2} \\
  &   &   &   &   &   & 1 &   &   &   &   \\
  &   &   &   &   &   &   & 1 &   &   &   \\
  &   &   &   &   &   &   &   & 1 &   &   \\
\pmb{a_2} &&&&& \pmb{a_1} &&&&& \pmb{a_0} \\
\end{pmatrix}
\quad
\begin{matrix}
00 \\ 01 \\ 02 \\ 10 \\ 11 \\ 12 \\ 20 \\ 21 \\ 22 
\end{matrix}
\end{equation}
Die komplexen Werte $a_0$, $a_1$, $a_2$ an den Stellen, die für x,t-Verschränkung stehen, rotieren ihre Plätze in den 3 betroffenen Spaltenvektoren und ebenfalls in den betroffenen Zeilenvektoren. Die Werte sind geeignet zu wählen, worauf wir noch kommen werden.

Hat der externe Beobachter zum Beispiel die Zeit t=0 gemessen, dann ist die Uhr im Zustand $\ket{00}_{ext}$. Dieser Zustand ist aus Sicht des internen Beobachters eine Verschränkung aus $\ket{00}_{int}$, $\ket{11}_{int}$ und $\ket{22}_{int}$.\footnote{Welche Bedeutung für ihn die Ziffernpaare haben, wissen wir nicht.} Mit den Bornschen Wahrscheinlichkeiten kollabiert die Überlagerung aus seiner Sicht:
\begin{equation*}
\begin{matrix}
\ket{00}_{ext} \ \xrightarrow{U} \ a_0\ket{00}_{int} + a_1\ket{11}_{int} + a_2\ket{22}_{int} 
& \xrightarrow{p = |a_0|^2} & \ket{00}_{int} \\ \\
& \xrightarrow{p = |a_1|^2} & \ket{11}_{int} \\ \\
& \xrightarrow{p = |a_2|^2} & \ket{22}_{int}
\end{matrix}
\end{equation*}
Damit das Geschehen weiterläuft, benötigen wir wieder den externen Beobachter. Die Inverse von $U$ transformiert zurück auf dessen Sicht und lässt unverschränkte interne Zustände $\ket{\chi\chi}_{int}$ verschränkt erscheinen.
\begin{equation}
U^{-1}\ =\ 
\begin{pmatrix}
\label{eq:not_U}
\pmb{a_0} &&&&& \pmb{a_1} &&&&& \pmb{a_2} \\
  & 1 &   &   &   &   &   &   &   &   &   \\
  &   & 1 &   &   &   &   &   &   &   &   \\
  &   &   & 1 &   &   &   &   &   &   &   \\
\pmb{a_2} &&&&& \pmb{a_0} &&&&& \pmb{a_1} \\
  &   &   &   &   &   & 1 &   &   &   &   \\
  &   &   &   &   &   &   & 1 &   &   &   \\
  &   &   &   &   &   &   &   & 1 &   &   \\
\pmb{a_1} &&&&& \pmb{a_2} &&&&& \pmb{a_0} \\
\end{pmatrix}
\quad
\begin{matrix}
00 \\ 01 \\ 02 \\ 10 \\ 11 \\ 12 \\ 20 \\ 21 \\ 22 
\end{matrix}
\end{equation}
Insgesamt haben wir nun einen stochastischen Prozess vorliegen. Die absolutquadrierten Elemente von $U$ und $U^{-1}$ liefern uns die Wahrscheinlichkeiten für Zustandsübergänge. Nur die fettgedruckten Elemente in (\ref{eq:U}) und (\ref{eq:not_U}) tragen zum Geschehen bei, wenn wir mit $\ket{\chi\chi}_{ext}$ Zuständen starten. Wir können die anderen Elemente somit weglassen und bekommen übersichtlichere 3x3-Matrizen. Wenn wir interne und externe Sicht noch in einem 6-komponentigen Vektor zusammenfassen, können wir eine \emph{unistochastische} Matrix $P$ angeben, die den Prozess beschreibt und von Bornschen Wahrscheinlichkeiten bevölkert ist.
\begin{equation}
P\, =\,
\begin{pmatrix}
0 & \{|U_{ji}|^2\} \\
\{|U_{ij}|^2\} & 0
\end{pmatrix}
\end{equation}
In unserem Beispiel ist die stochastische Matrix, wenn wir noch $|a_i|^2$ durch $p_i$ abkürzen
\begin{equation}
P\, =\,
\begin{pmatrix}
&&& p_0 & p_1 & p_2 \\
&&& p_2 & p_0 & p_1 \\
&&& p_1 & p_2 & p_0 \\
p_0 & p_2 & p_1 &&& \\
p_1 & p_0 & p_2 &&& \\
p_2 & p_1 & p_0 &&& 
\end{pmatrix}
\quad
\begin{matrix}
00_{ext} \\ 11_{ext} \\ 22_{ext} \\ 00_{int} \\ 11_{int} \\ 22_{int}
\end{matrix}
\end{equation}
Die Matrix $P^2$ bedeutet einen Wechsel auf die interne Sicht und wieder zurück. Sie steht quasi für einen Tick der Zeit und liefert die Wahrscheinlichkeiten, mit denen der externe Beobachter einen bestimmten x,t-verschränkten Zustand nach dem Tick beobachtet in Abhängigkeit des Zustands vor dem Tick.

Wir wählen nun konkrete Zahlenwerte für die Uhr. Im Anhang wird gezeigt, wie man auf geeignete Werte kommt.
\begin{equation*}
a_0=N\frac{sin(\alpha_0+\alpha_1)}{sin(\alpha_1-2\alpha_0)}e^{i\alpha_0} \quad
a_1=N\frac{sin(\alpha_0+\alpha_1)}{sin(\alpha_0-2\alpha_1)}e^{i\alpha_1} \quad
a_2=N
\end{equation*}
mit 
\begin{equation*}
N\, = \, \left( 1 +
\frac{sin^2(\alpha_0+\alpha_1)}{sin^2(\alpha_1-2\alpha_0)} +
\frac{sin^2(\alpha_0+\alpha_1)}{sin^2(\alpha_0-2\alpha_1)} \right)^{-\frac{1}{2}}
\end{equation*}
Und wir wählen die Winkel zu
\begin{equation*}
\alpha_0 = \pi / 6 \quad \alpha_1 = \pi / 5 \\ \\
\end{equation*}
Damit werden 
\begin{equation*}
P\, \approx\,
\begin{pmatrix}
   &&& 0.63787 &  0.12644 &  0.23569 \\
   &&& 0.23569 &  0.63787 &  0.12644 \\
   &&& 0.12644 &  0.23569 &  0.63787 \\
   0.63787 &  0.12644 &  0.23569 &&& \\
   0.23569 &  0.63787 &  0.12644 &&& \\
   0.12644 &  0.23569 &  0.63787 &&& 
\end{pmatrix}
\quad
\begin{matrix}
00_{ext} \\ 11_{ext} \\ 22_{ext} \\ 00_{int} \\ 11_{int} \\ 22_{int}
\end{matrix}
\end{equation*}
und
\begin{equation*}
P^2\, \approx\,
\begin{pmatrix}
   0.47841 & 0.26079 & 0.26079 &&& \\
   0.26079 & 0.47841 & 0.26079 &&& \\
   0.26079 & 0.26079 & 0.47841 &&& \\
   &&& 0.47841 & 0.26079 & 0.26079 \\
   &&& 0.26079 & 0.47841 & 0.26079 \\
   &&& 0.26079 & 0.26079 & 0.47841
\end{pmatrix}
\quad
\begin{matrix}
00_{ext} \\ 11_{ext} \\ 22_{ext} \\ 00_{int} \\ 11_{int} \\ 22_{int}
\end{matrix}
\end{equation*}
Erwartungswerte für die beobachtete Zeit können wir so bilden
\begin{equation*}
<t> = t \cdot Q \cdot \psi \quad\quad t = \begin{pmatrix}
0&1&2&0&0&0
\end{pmatrix}
\end{equation*}
In unserem Beispiel ergeben sich in Abhängigkeit des Ausgangszustands die Erwartungswerte
\begin{center}
\begin{tabular}{ |c|c| } 
 \hline
 $\psi$ & $<t>$ \\ 
 \hline
 $\ket{00}_{ext}$ & 0.78 \\ 
 $\ket{11}_{ext}$ & 1.00 \\ 
 $\ket{22}_{ext}$ & 1.22 \\
 \hline
\end{tabular}
\end{center}
Eine makroskopische Uhr, die über viele Quantenuhren mittelt, würde uns diese Zeitwerte liefern.
Ausgehend von t=0 liefe die gemessene Zeit also vorwärts, bei t=1 bliebe sie stehen, ausgehend von t=2 liefe sie rückwärts. Könnten wir mit sehr großen Matrizen $U$ hantieren, dann könnten wir womöglich große Bereiche schaffen, in denen die Zeit vorwärts liefe und solche, in denen sie rückwärts liefe. In jedem solchen Bereich erführe der externe Beobachter aber \emph{im Mittel eine in eine bestimmte Richtung laufende Zeit}.

Größere doppelt-stochastische Matrizen mögen sich zu diesem Zweck noch leicht zurechtlegen lassen. Allerdings ist die Frage, ob sie auch unistochastisch sind, ab einer Zeilenzahl von 5 und dem heutigen Stand der Forschung leider kaum beantwortbar. Unistochastisch müssen sie aber sein, wenn sie mit der Quantenmechanik verträglich sein sollen.

\section{Das Anwachsen der Entropie in der psychischen Zeit}

Doppelt-stochastische Übergangsmatrizen führen im Limes zu einer Gleichverteilung der Wahrscheinlichkeiten.

%The stationary distribution of an irreducible aperiodic finite Markov chain is uniform if and only if its transition matrix is doubly stochastic.

\section{Eine relativbewegte Quantenuhr}

Wir fragen nun danach, wie eine bewegte Uhr gesehen wird. Wenn der externe Beobachter die Zeigerstellung $\chi$ erkennt, dann schließt er daraus, dass in der bewegten Uhr die Zeit $\chi$ vergangen ist. Daran ändert eine Relativbewegung der Uhr nichts. Das heißt, eine Relativbewegung soll nichts daran ändern, welche Ortseigenvektoren mit welchen Zeiteigenvektoren verschränkt sind. Im diskreten Fall kann sich also an der Verschränkung gar nichts ändern. Im Raumzeitkontinuum gibt es dagegen mehr Freiheit. Wir greifen zurück auf den kontinuierlichen Uhrenzustand (\ref{eq:psi_clock}). 

Eine Koordinatentransformation der internen Uhrkoordinaten $x,t$ auf die Koordinaten $x',t'$ aus Sicht des externen Beobachters soll also bewirken
\begin{equation} 
\psi(x,t) \sim \delta(x-t)\ \longmapsto \ \psi(x',t') \sim \delta(x'-t')
\end{equation} 
Bemerkenswerterweise ist ein Lorentz-Boost in x-Richtung solch eine verschränkungserhaltende Transformation, denn
\begin{equation} 
\delta(x'-t') =\, \delta(\gamma (x + \beta t) - \gamma (t + \beta x)) 
= \frac{\delta(x-t)}{\gamma (1-\beta)} \sim \delta(x-t)
\end{equation}
wenn wir t in Metern messen, was wir schon von Anfang an getan haben. Allerdings trifft das für Uhrenzeigerkoordinaten in y- oder z-Richtung nicht zu, da bei einem Boost in x-Richtung $y'=y$ und $z'=z$ gilt.

\section{Anhang: Berechnung der Matrixelemente von U}

Wir kümmern uns hier nur um die $a_i$ und stauchen U auf eine 3x3-Matrix zusammen. Ohne Beschränkung der Allgemeinheit kann eines der Elemente reell gewählt werden. Wir machen den Ansatz
\begin{equation*}
U\, =\, N \, 
\begin{pmatrix}
r_0 e^{i\alpha_0} & 1 & r_1 e^{i\alpha_1} \\
r_1 e^{i\alpha_1} & r_0 e^{i\alpha_0} & 1 \\
1 & r_1 e^{i\alpha_1} & r_0 e^{i\alpha_0}
\end{pmatrix}
\quad N, r_i, \alpha_i \in \mathbb{R}
\end{equation*}
$N$ ist eine noch zu bestimmende Normierungskonstante. Damit die Spaltenvektoren paarweise orthogonal sind, muss gelten
\begin{equation*}
r_0 e^{i\alpha_0} + r_0 r_1 e^{i(\alpha_1 - \alpha_0)} + r_1 e^{-i\alpha_1} = 0
\end{equation*}
und damit 
\begin{equation*}
r_0 = - \frac{r_1 e^{-i\alpha_1}}{ e^{i\alpha_0} + r_1 e^{i(\alpha_1 - \alpha_0)} }
\end{equation*}
Gemäß Voraussetzung muss diese Größe reell sein. Wenn wir mit dem konjungiert komplexen Nenner erweitern wird der Nenner reell und der Zähler wird zu
\begin{equation*}
-r_1 \left( e^{-i(\alpha_0+\alpha_1)} + r_1 e^{i(\alpha_0-2\alpha_1)} \right)
\end{equation*}
Damit dieser Term reell wird, muss sein
\begin{equation*}
r_1=\frac{sin(\alpha_0+\alpha_1)}{sin(\alpha_0-2\alpha_1)}
\end{equation*}
Entsprechend kommt man auf 
\begin{equation*}
r_0=\frac{sin(\alpha_0+\alpha_1)}{sin(\alpha_1-2\alpha_0)}
\end{equation*}
und damit für die Normierungskonstante auf 
\begin{equation*}
N\, = \, \left( 1 +
\frac{sin^2(\alpha_0+\alpha_1)}{sin^2(\alpha_1-2\alpha_0)} +
\frac{sin^2(\alpha_0+\alpha_1)}{sin^2(\alpha_0-2\alpha_1)} \right)^{-\frac{1}{2}}
\end{equation*}
\end{document}
