\documentclass[12pt]{book}
\usepackage[margin=2cm, bindingoffset=0cm]{geometry}
%\usepackage[german]{babel} 
\usepackage[ngerman]{babel} %sudo apt-get install texlive-lang-german
\usepackage[parfill]{parskip}
\usepackage[utf8]{inputenc}
\usepackage[dvipsnames]{xcolor}
\usepackage{tcolorbox}
\usepackage{helvet} 
\usepackage{framed}
\usepackage{anyfontsize}
\definecolor{quotecolor}{rgb}{0.8,0.9,1}
\renewcommand{\familydefault}{\sfdefault} 
\setlength\parindent{0pt}
\tcbset{width=0.9\textwidth,boxrule=0pt,colback=quotecolor,arc=0pt,auto outer arc,left=0pt,right=0pt,boxsep=5pt}

\begin{document}
\title{\fontsize{40}{40}\selectfont \textbf{Die Schrift}}
\date{\today}
\maketitle

\chapter{Wozu diese Schrift?}

Diese Schrift soll dich verändern. Jede Schrift, die du liest, verändert dich. 
Ich will, dass diese Schrift dein Denken in eine bestimmte Richtung verändert.
Diese Schrift ist nicht dazu da, mir zum Broterwerb zu dienen.
Es geht einzig und allein darum, dich zu verändern. Von solcher Art ist diese Schrift. 

Wenn du meinst, dass du dich nicht ändern solltest, zum Beispiel weil du dich
bereits perfekt findest, dann kehre jetzt um und lies nicht weiter!

\section{Warum will ich dein Denken ändern?}

Die kurze Antwort ist: aus Mitleid. 

Wenn du meinst, dass du kein Mitleid benötigst, dann warte ab und 
schaue, was dir das Leben bringen wird! Und danach komme wieder, falls du noch kannst!

Die lange Antwort ist: Du wirst sterben. Du wirst leiden. Du wirst nicht verstehen, warum du leiden musst.
Während du lebst, wirst du Vielen unnötiges Leid zufügen. Das wirst du teilweise absichtlich machen, viel öfter 
wirst du es aber unabsichtlich machen. Diese Schrift soll dir helfen zu erkennen, dass du
weder deinem eigenen Leiden noch dem Leidzufügen aus dem Wege gehen kannst.
Das ist einem Menschen nicht möglich. Dennoch kannst du versuchen, diese Leiden zu verringern.

Vielleicht gehörst du zu den unglücklichen Tröpfen, die sich ständig um das Morgen sorgen.
Ich sage dir: du brauchst dich nicht sorgen, denn morgen kannst du schon tot sein. Vielleicht sogar heute schon.

Oder haderst du mit der Welt? Geht sie in eine schlechte Richtung?
Ich sage dir: weder kannst du wissen, was gut oder schlecht ist, noch in welche Richtung sie gehen wird.
Du kannst nicht mal wissen, ob sie überhaupt geht. Vielleicht bist nur du es, der geht.

Hast du vielleicht Angst vor dem Tod? Du sollst verstehen, warum du diese Angst haben musst.
Doch vielleicht weißt du nicht, was der Tod ist. Vielleicht weißt du nicht mal, was Leben ist?

\section{In welche Richtung will ich dein Denken ändern?}

Dies ist eine philosophische Schrift. Es geht um Erkenntnis. Wenn du als naiver Realist hier aufschlägst, soll diese Schrift
dein Denken dramatisch umdrehen. Niemals würdest du dich als naiv bezeichnen?

Wenn du diese Sätze verstehst, dann bist du kein naiver Realist mehr, und diese Schrift kann dir nicht viel Neues geben:
\begin{quote}\begin{tcolorbox}
Jedes Elementarteilchen enthält alle anderen Elementarteilchen.
\end{tcolorbox}\end{quote}
\begin{quote}\begin{tcolorbox}
There are no particles!
\end{tcolorbox}\end{quote}
\begin{quote}\begin{tcolorbox}
Die Summe zweier Teile ist ihr Produkt. Im Produkt sind die Teile enthalten und nicht enthalten. 
Auch unendlich viele andere 2 Teile sind im Produkt enthalten und nicht enthalten.
\end{tcolorbox}\end{quote}
\begin{quote}\begin{tcolorbox}
Keine Nachricht enthält eine Bedeutung. Auch diese Schrift enthält keine Bedeutung.
\end{tcolorbox}\end{quote}
\begin{quote}\begin{tcolorbox}
Bei diesen kleinsten Lebewesen aber wird die Frage, ob sie aus lebendiger oder toter Materie bestehen, unentscheidbar. Man kann dies so ausdrücken, daß es überhaupt nur lebendige Materie gebe; ...
\end{tcolorbox}\end{quote}
\begin{quote}\begin{tcolorbox}
Eigentlich gibt es gar nichts Unlebendiges! ...
\end{tcolorbox}\end{quote}
\begin{quote}\begin{tcolorbox}
Dass ein Tisch im Grunde auch lebendig ist, bemerken wir nicht, weil wir ihn nur vergröbert betrachten und damit vereinfacht sehen.
\end{tcolorbox}\end{quote}

Wenn du als Materialist hier aufschlägst, sollst du als Idealist wieder herausgehen.

Hinter dieser Schrift steckt der feste Glaube: wenn du mehr erkennst von der Welt und dir selbst,
dann wird sich über kurz oder lang dein Handeln von selbst in die gewollte Richtung ändern. Du wirst weniger wollen, mit weniger zufrieden
oder gar glücklich sein. Du wirst Leid vermeiden wollen. Du wirst dein Handeln von der Liebe leiten lassen, weil du weißt,
dass es sonst nichts gibt, an dem du dich festhalten könntest.

\section{Wie werden wir es tun?}

Wir beginnen mit dem philosophischen Fundament von Descartes.
Wir erkennen die Schleier, die unser Denken vernebeln und uns an der Erkenntnis hindern.
Wir sehen nach, was die Wissenschaft uns seit den alten Zeiten der Philosophie Neues gebracht hat.
Wir gehen die harte Tour, auch mit Mathematik, weil die Sprache der Mathematik näher an der Wahrheit ist als unsere Alltagssprache, welche einer der Schleier ist.

\chapter{Philosophischer Startpunkt}

\section{cogito ergo sum}

\section{Die Schleier}

\end{document}