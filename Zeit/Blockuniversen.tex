\documentclass[12pt]{article}
\usepackage[margin=2cm, bindingoffset=0cm]{geometry}
%\usepackage[german]{babel} 
\usepackage[ngerman]{babel} %sudo apt-get install texlive-lang-german
\usepackage{hyperref} % web links etc.
\usepackage[parfill]{parskip}
\usepackage[utf8]{inputenc}
\usepackage[dvipsnames]{xcolor}
\usepackage{tcolorbox}
\usepackage{helvet} 
\usepackage{framed}
\usepackage{anyfontsize}
\usepackage{csquotes}
\usepackage{mathrsfs}
\usepackage{physics}
\setcounter{MaxMatrixCols}{16}
\usepackage{amssymb}
%\usepackage{MnSymbol}
\usepackage{leftidx}
\MakeOuterQuote{"}
\definecolor{quotecolor}{rgb}{0.8,0.9,1}
\renewcommand{\familydefault}{\sfdefault} 
\setlength\parindent{0pt}
\tcbset{boxrule=0pt,colback=quotecolor,arc=5pt,auto outer arc,left=5pt,right=5pt,boxsep=5pt}
%  width=0.9\textwidth,

\setcounter{secnumdepth}{-1}
\begin{document}
\title{\fontsize{25}{25}\selectfont \textbf{Eine Abfolge von Blockuniversen}}
\author{Harald Rieder}
\date{\today}
\maketitle

%\begin{abstract}

%\end{abstract}

\tableofcontents

\section{Motivation}

Das \href{https://en.wikipedia.org/wiki/Measurement_problem}{Messproblem} der Quantenmechanik entsteht aus der Vorstellung einer stetigen Entwicklung von mathematischen Größen in der als absolut angesehen gemeinsamen Zeit in Verbindung mit der Erfahrung, dass Kenntnis über das durch die mathematischen Größen beschriebene Quantending\footnote{Bewusst vermeiden wir den Begriff "Quantensystem". Ein System besteht aus primär existierenden Teilen, die sekundär miteinander wechselwirken. Das ist eine Vorstellung, die der Quantenwelt nicht gerecht wird. Die Teilung eines Hilbertraums in Unterräume ist immer irgendwo willkürlich.} nur dann erlangt werden kann, wenn sich dabei dessen Zustand unstetig ändert. Dadurch sind zu jedem Zeitpunkt 2 verschiedene sich daran anschließende Arten von Zukünften möglich und niemand kann sagen, wann oder warum die eine oder die andere gewählt wird. 

Dieses Paradoxon hat schon die besten Köpfe herausgefordert und der ein oder andere dachte sich, eine Lösung zu haben. Mitunter taucht die Denke auf, dass verschiedene Sichten auf die Quantenmechanik möglich seien, so genannte "Interpretationen der Quantenmechanik", die am Ende doch nur eine Frage des Geschmacks seien. Wir werden hier diesen Weg nicht gehen. Statt dessen soll das Paradoxon dadurch aufgelöst werden, dass zu jedem Zeitpunkt nur eine Art von Zukunft möglich ist. Anstatt an der Vorstellung eines Zeitparameters, von dem Geschehen stetig abhängt, festzuhalten, werfen wir diese Vorstellung komplett über Bord. Es gibt einige gute Gründe, es auf diese Art zu versuchen:

\begin{enumerate}
    \item Unterhalb der Planck-Zeit kann es keine stetige Zeitentwicklung mehr geben.
    \item Eine Propagation des Geschehens durch lineare Operatoren, wie sie für die sich stetig anschließende Zukunft im Modell geschieht, treibt das Weltgeschehen eben \emph{linear} weiter. Man kann schwerlich glauben, dass dadurch irgend etwas wie Lebendigkeit erfolgreich modelliert werden kann. Aus der klassischen Chaos-Theorie haben wir gelernt, dass erst eine nichtlineare Dynamik zu etwas führt, dass uns an Leben erinnert. Die unstetig sich anschließende Zukunft liefert uns so eine nichtlineare Dynamik.
    \item Um Kenntnis über ein Quantending zu erlangen ist ein Verlassen der stetigen Entwicklung irgendwo\footnote{an einem sogenannten Heisenberg-Schnitt} auf dem langen Weg in das Bewusstsein notwendig. Die Dekohärenztheorie kann erklären, wie Information aus einem Quantending in seine Umgebung fließt unter der Annahme einer gemeinsamen stetigen Entwicklung. Doch wenn Kenntnis über Quantending (und seine Umgebung) erlangt werden soll, ist ein unstetiger Schritt notwendig. Das \emph{problem of outcomes} wird durch die Dekoränztheorie nicht gelöst, und das ist das eigentliche Paradoxon.
    \item Eine komplett neue Dynamik könnte erfunden werden. Das erscheint aber schwieriger, als einfach die Hälfte der Dynamik wegzuwerfen.
\end{enumerate}
Die Quantenmechanik liefert uns statistische Größen: Wahrscheinlichkeiten für Ereignisse. Wir haben gelernt, wie man Empfindungen mit den Ereignissen zu verknüpfen hat, die das mathematische Modell hergibt. Dabei hat sich herausgestellt, dass diese Theorie Häufigkeiten und Mittelwerte teilweise erschreckend genau liefern kann. Wenn die Hälfte der Dynamik über Bord geworfen werden soll, dann müssen dennoch die statistischen Aussagen der Theorie im experimentell gesteckten Rahmen erhalten bleiben. Ob dies überhaupt möglich ist, soll hier untersucht werden. 

Die unstetige Dynamik hat die Eigenart, dass das Geschehen zum Erliegen kommt, wenn man immer nach derselben Information fragt. In technischer Ausdrucksweise führt die Anwendung eines projektiven Messoperators zum "Kollaps" des Zustandsvektors auf einen Eigenraum des Messoperators. Danach führt die Wiederholung der projektiven Messung zu keinem Verlassen des Eigenraums mehr. Um ein Geschehen mit der unstetigen Dynamik für einen Beobachter scheinbar am Laufen zu halten, muss es Prozessschritte geben, die ständig neue Überlagerungen herstellen.

Während in gewöhnlicher Quantentheorie mit der Beobachtung üblicherweise Schluss ist, denn dann können die subjektiven Empfindungen gegen die Zahlen des mathematischen Modells verglichen werden, fordern wir ständige Beobachtungen, damit ein Geschehen überhaupt stattfinden kann. Da das menschliche Subjekt aber nur mit der Seite der projektiven Messungen in Verbindung steht, ist ihm die andere verborgen. Natürlich wäre es ontologisch das Einfachste anzunehmen, dass die anderen Prozessschritte genauso wie die einen stattfinden, dass sie am Ende mit Empfindungen in Zusammenhang stehen, nur eben für andere Subjekte. 

\section{Grundlagen}

\subsection{Koordinatentransformationen, Produkträume und Verschränkung}

Zu tun: 

Indextransformationen, bei denen sich die Zahl der Indexe nicht ändert, entsprechen im Kontinuum Koordinatentransformationen. 

Bei Bildung von Produkträumen lassen sich die Indexe der Einzelräume auf einen einzigen Produktraumindex bijektiv abbilden. Wichtig: dies gilt auch für das Kontinuum! Auch dort existieren immer bijektive Abbildungen aufgrund der Gleichmächtigkeit der beteiligten Mengen.

Transformationen, die über Teilräume hinüberreichen, ändern i.A. deren Verschränkung. Solche, die sich nur auf Teilräume beschränken, erhalten die von-Neumann-Entropie.

\subsection{Orts-, Impuls-, Zeit- und Energieoperatoren}

Weil wir ihnen später begegnen werden, rufen wir uns die Matrixelemente spezieller Operatoren und die Komponenten ihrer Eigenvektoren in speziellen Vektorbasen in Erinnerung. Wir unterscheiden nicht zwischen abzählbaren und kontinuierlichen Dirac-Vektoren und verwenden durchgängig die Indexschreibweise, d.h. $\{\psi_{x_j}\} \equiv \psi(x)$: aus unendlich vielen durch $x_j$ indizierten Vektorkomponenten entsteht im Grenzfall deren unendlicher Dichtheit eine Funktion einer kontinuierlichen Variable $x$. 

Zur Notation: Im Kontinuum stellt die Variable $x$ einen kontinuierlichen Index dar. In Integralen treten Terme wie $\mathrm{d}x$ auf, wir haben dort also $\int\mathrm{d}x$. Dem entspricht im Diskreten eine Summe $\sum_{x_j}$. Gewöhnlich wird in der Literatur nur der Index $j$ hingeschrieben, also $\sum_j$ und aus dem jeweiligen Zusammenhang geht hervor, welche Basis der Index $j$ indiziert. Zwecks größerer Klarheit werden wir Vektorkomponenten meist mit $x_j$ usw. indizieren.

Die Matrixelemente des Ortsoperator $\hat{x}$ und die Komponenten seiner Eigenvektoren $\ket{x_j}$ zu den Eigenwerten $x_j$ sind in der Ortsbasis $\{\,\ket{x_j}\,\}$
\begin{equation}
\bra{x_j}\ket{\,\hat{x}\,x_k}\ =\ x_{x_j x_k}\ =\ \delta_{x_j x_k}\,x_j
\quad\quad 
\bra{x_k}\ket{x_j}\ =\ \psi_{x_j x_k}\ =\ \delta_{jk}
\end{equation}

und in der Impulsbasis\footnote{Da wir weiter unten einen Zeitimpuls einführen, müssten wir hier eigentlich genauer von einem Ortsimpuls sprechen.} $\{\,\ket{p_j}\,\}$
\begin{equation}
\bra{p_j}\ket{\,\hat{x}\,p_k}\ =\ x_{p_j p_k}\ =\ 
-\mathrm{i}\hbar\,\frac{\partial}{\partial p_k}\,\delta_{p_j p_k}
\quad\quad 
\bra{p_k}\ket{x_j}\ =\ \psi_{x_j p_k}\ =\ 
\frac{1}{\sqrt{2\pi}} e^{-\frac{i}{\hbar}x_j p_k}
\end{equation}
$\delta_{x_j x_k}$ soll im Kontinuum dem $\delta(x_j - x_k)$ entsprechen.

Wolfgang Pauli schrieb in \emph{Die allgemeinen Prinzipien der Wellenmechanik}: "Wir schließen also, daß auf die Einführung eines Operators $\hat{t}$ grundsätzlich verzichtet und die Zeit $t$ in der Wellenmechanik notwendig als gewöhnliche Zahl betrachtet werden muß." Dies mag für die nicht-relativistische Quantenmechanik praktikabel gewesen sein. Relativistische Theorien legen jedoch die Gleichbehandlung von Raum und Zeit nahe, und dies wollen wir hier auch tun.

Die Matrixelemente des Zeitoperators $\hat{t}$ und die Komponenten seiner Eigenvektoren $\ket{t_j}$ zu den Eigenwerten $t_j$ sind in der Zeitbasis $\{\,\ket{t_j}\,\}$
\begin{equation}
\bra{t_j}\ket{\,\hat{t}\,t_k}\ =\ t_{t_j t_k}\ =\ \delta_{jk}\,t_j
\quad\quad 
\bra{t_k}\ket{t_j}\ =\ \psi_{t_j t_k}\ =\ \delta_{jk}
\end{equation}

und in der (Zeit-)Impuls- oder Energiebasis $\{\,\ket{E_j}\,\}$
\begin{equation}
\label{eq:time_components}
\bra{E_j}\ket{\,\hat{t}\,E_k}\ =\ t_{E_j E_k}\ =\ 
-\mathrm{i}\hbar\,\delta_{jk}\,\frac{\partial}{\partial E_k}
\quad\quad 
\bra{E_k}\ket{t_j}\ =\ \psi_{t_j E_k}\ =\ 
\frac{1}{\sqrt{2\pi}} e^{\frac{i}{\hbar}t_j E_k}
\end{equation}

\subsection{Die Bornsche Regel}

In der ursprünglichen (1926) Formulierung besagte die \href{https://en.wikipedia.org/wiki/Born_rule}{Bornsche Regel}, dass 
\begin{itemize}
\item bei einer Messung einer Observable einer der Eigenwerte $\{\lambda_j\}$ des zugehörigen Messoperators $\hat{L}$ beobachtet wird.
\item die Wahrscheinlichkeit für die Messung des Eigenwerts $\lambda_j$ sich aus dem Absolutquadrat des Skalarprodukts zwischen Anfangszustand $\ket{\psi^{(0)}}$ und Eigenvektor $\ket{\lambda_j}$ bestimmt.
\end{itemize}
Nach der Messung liegt der Eigenzustand $\ket{\psi^{(1)}} = \ket{\lambda_j}$ vor und bei wiederholter Messung von $\hat{L}$ ändert sich daran nichts mehr. 

Offen bleiben musste zunächst, was genau eigentlich eine Messung ist. Das Messexperiment wurde in der Sprache der klassischen Physik beschrieben. Man hatte ein paar mechanische Größen wie Ort und Impuls, die sich in die Sprache der Quantenmechanik übertragen ließen. Irgendwo musste es einen Übergang zwischen Quantenmechanik und klassischer Physik geben, doch wo und wie war unklar.

An dieser Lage haben sich ein paar Dinge in 100 Jahren geändert:
\begin{itemize}
\item In den Gleichunge der Quantentheorien gibt es Größen, die kein klassisches Analogon haben, z.B. den Spin. 
\item Die meisten Physiker dürften daran glauben, dass die Natur vollständig quantenmechanisch zu beschreiben ist, und dass die klassische Physik irgendwie daraus abzuleiten ist.
\item Mit der Dekoährenztheorie ist es mit quantenmechanischen Mitteln gelungen, mehr von dem zu beschreiben, was bei einer Messung geschieht.
\end{itemize}

Heute sollten wir uns die Messung vorstellen als einen Prozess, der ein zu beobachtendes Quantending zunächst quantendynamisch an die Umgebung koppelt. 
Dabei entsteht bei einer idealen Messung ein Zustand, der einerseits immer noch eine Überlagerung mit den ursprünglichen Amplituden $\psi^{(0)}_{\lambda_j}$ bezogen auf die Messbasis $\{\,\ket{\lambda_j}\,\}$ darstellt, andererseits die Messbasis maximal verschränkt mit einer entsprechenden Umgebungsbasis $\{\,\ket{e_j}\,\}$. Die Verschränkung soll nach der Dekohärenztheorie mittels eines in der Zeit stetigen deterministischen Prozesses $\hat{U}(t-t_0)$ erfolgen.
\begin{equation}
\label{eq:decoherence}
\ket{e^{(0)}}\otimes\ket{\psi^{(0)}}\ =\ \ket{e^{(0)}}\otimes\sum_j \psi^{(0)}_{\lambda_j} \ket{\lambda_j}
\ \xrightarrow{\hat{U}(t_1-t_0)}\ \sum_j \psi^{(0)}_{\lambda_j} \left(\, \ket{e_j} \otimes \ket{\lambda_j}\, \right)
\end{equation}
Dadurch ist Information über $\ket{\psi^{(0)}}$ in der Umgebung verfügbar geworden und kann abgefragt werden. Dieser finale Abfrageschritt ist Stand heute nicht verstanden. Es ist aber gewiss, dass der Endzustand in \eqref{eq:decoherence} im Allgemeinen nicht bewusst erfahren werden kann, sondern nur Zustände nach bestimmten finalen Schritten, zum Beispiel Projektionen auf die Ortsbasis\footnote{Dies steht im Zusammenhang mit der Frage, warum noch keine Schrödinger-Katzen beobachtet wurden.}. Eventuell ist es auch nur der Abfrageschritt selbst, der bewusst erfahren wird. Der Abfrageschritt ist verbunden mit einem Kollaps der Überlagerung in einen der Endzustände mit der Bornschen Wahrscheinlichkeit.
\begin{equation} 
\label{eq:collapse}
\sum_j \psi^{(0)}_{\lambda_j} \left( \ket{e_j} \otimes \ket{\lambda_j} \right)
\ \xrightarrow{p_j^{(0)\rightarrow(1)}\ =\ \left|\psi^{(0)}_{\lambda_j}\right|^2}\ 
\ket{e_j} \otimes \ket{\lambda_j}
\end{equation}
$\left|\psi^{(0)}_{\lambda_j}\right|^2$ ist die Übergangswahrscheinlichkeit 
\begin{equation} 
\label{eq:orig_Born}
p^{(0)\rightarrow(1)} 
= \Bigl| \bra{\lambda_j}\ket{\psi^{(0)}} \Bigr| ^2
=\ \Bigl| \bra{\psi^{(1)}}\ket{\psi^{(0)}} \Bigr| ^2
\end{equation}
in der ursprünglichen Formulierung der Bornschen Regel.

Ob nun \eqref{eq:decoherence} tatsächlich stattfindet ist von einem pragmatischen Standpunkt aus gesehen unerheblich\footnote{"shut up and calculate"}, so lange sich der Dekohärenzprozess der Beobachtbarkeit aufgrund fehlender technologischer Fähigkeiten entzieht. Beobachtete Ereignisse mit ihren zugehörigen Wahrscheinlichkeiten sind am Ende dieselben. Doch seit einigen Jahren sind Dekohärenzprozesse technologisch greifbar geworden, z.B. in einem Experiment von \href{https://journals.aps.org/prl/abstract/10.1103/PhysRevLett.98.200402}{Sonnentag und Hasselbach} aus dem Jahr 2006.

Nimmt man an, dass erstens der Beobachter Information nicht direkt aus dem Quantenteil abgreifen kann, sondern nur indirekt über dessen Umgebung, und zweitens die Quantentheorie am Ende der gesamten klassischen Physik zugrunde liegen muss, dann drängt sich ein Prozess \eqref{eq:decoherence} aus logischen Gründen geradezu auf. Die theoretische Idee reicht \href{https://de.wikipedia.org/wiki/Dieter_Zeh}{in das Jahr 1970 zurück}.
 

\section{Die Zeit in der Quantenmechanik}
\subsection{Zerlegung in H- und E-Raum}
Gleichungen der Quantenmechanik sind vom Prinzip her eigentlich einfach. Ein linearer Operator $\hat{O}$ wird angewendet auf einen Zustandsvektor $\ket{\psi}$ und das Ergebnis soll $0$ ergeben.
\begin{equation*} 
\hat{O}\ket{\psi} = 0 
\end{equation*}
Beim Blick auf Schrödinger-, Dirac- und Klein-Gordon-Gleichungen fällt eine Gemeinsamkeit in's Auge. Der Operator $\hat{O}$ wirkt immer in einem Produktraum $\mathscr{H} \otimes \mathscr{E}$. Er hat die Gestalt 
\begin{equation*} 
\hat{O}\ =\ ^\mathscr{H}\hat{H}\ \otimes\ ^\mathscr{E}\hat{1}\ -\ ^\mathscr{H}\hat{1}\ \otimes\ ^\mathscr{E}\hat{E}
\end{equation*}
und ist damit die Summe aus einem Operator $\hat{H}$, der nur im Teilraum $\mathscr{H}$ wirkt und einem Operator $\hat{E}$, der nur im Teilraum $\mathscr{E}$ wirkt. Der Teilraum $\mathscr{E}$ ist der Zeitunterraum und eine wählbare Basis ist die der Zeiteigenvektoren $\ket{t_j}$. Der Teilraum $\mathscr{H}$ enthält alles andere.

Durch die Zerlegung bietet sich ein Produktansatz zur Lösung dieser Gleichungen an. Mit
\begin{equation*} 
\ket{\psi}\ =\ ^\mathscr{H}\ket{\psi}\ \otimes\ ^\mathscr{E}\ket{\psi}
\end{equation*}
bekommen wir
\begin{equation} 
\label{eq:productspace}
\hat{H}\ ^\mathscr{H}\ket{\psi}\ \otimes\ ^\mathscr{E}\ket{\psi}
\ - \ 
^\mathscr{H}\ket{\psi}\ \otimes\ \hat{E}\ ^\mathscr{E}\ket{\psi} \ =\ 0
\end{equation}
%Funktioniert für Schrödinger- und Dirac-Gleichung. Bsp. liefern...
und die Gleichung zerfällt in 2 Eigenwertgleichungen für die Operatoren $\hat{H}$ und $\hat{E}$.
\begin{equation}
\label{eq:system}
\begin{split}
\hat{H}\ ^\mathscr{H}\ket{\psi} = E\ ^\mathscr{H}\ket{\psi}\\
\hat{E}\ ^\mathscr{E}\ket{\psi} = E\ ^\mathscr{E}\ket{\psi}\ 
\end{split}
\end{equation}
Dieses Gleichungssystem ist noch gekoppelt über die Eigenwerte $E$. 

Aufgrund der Linearität von $\hat{H}$ und $\hat{E}$ sind alle Linearkombinationen von Produkten zum selben Eigenwert $E$ ebenfalls Lösungen. Die allgemeine Lösung
\begin{equation}
\label{eq:general_solution}
\ket{\psi}\ =\ \sum_j\ \psi_j\ ^\mathscr{H}\ket{E_j}\ \otimes\ ^\mathscr{E}\ket{E_j}
\end{equation}
mit mehr als einem Summand steht für eine (natürlich subjektive) Verschränkung der Teilräume $\mathscr{H}$ und $\mathscr{E}$. Die Darstellung \eqref{eq:general_solution} von $\ket{\psi}$ ist eine \href{https://en.wikipedia.org/wiki/Schmidt_decomposition}{Schmidt-Zerlegung}.

Die unverschränkten Lösungen $^\mathscr{H}\ket{E_j}\ \otimes\ ^\mathscr{E}\ket{E_j}$ werden üblicherweise als \emph{stationäre Zustände} oder \emph{Energieeigenzustände} bezeichnet. Denn der Differentialoperator $\mathrm{i}\hbar\,\frac{\partial}{\partial t}$, der in den Gleichungen auftritt, muss aufgefasst werden als Matrixelemente $E(t-t')$ des kontinuierlichen Energieoperators $\hat{E}$ in der Zeitdarstellung. Nähme man es genauer, müsste man immer noch die Delta-Distribution dazuschreiben: 
\begin{equation*}
E(t-t')\ =\ \mathrm{i}\hbar\,\frac{\partial}{\partial t}\,\delta(t-t')
\end{equation*}

Der gesamte Produktraum $\mathscr{H} \otimes \mathscr{E}$ kann aus den allgemeinen Produkten der Eigenvektoren von $\hat{H}$ und $\hat{E}$ aufgespannt werden:
\begin{equation}
\label{eq:general_state}
\ket{\psi}\ =\ \sum_{jk}\ \psi_{jk}\ ^\mathscr{H}\ket{E_j}\ \otimes\ ^\mathscr{E}\ket{E_k}
\end{equation}
Doch alle Zustände, die Terme mit $j \neq k$ haben, sollen "in der Natur nicht vorkommen". Etwas geringer könnte man auch fordern, dass sie durch die uns zur Verfügung stehenden Messprozesse nicht beobachtbar sind.

Die Bornschen Übergangswahrscheinlichkeiten zwischen Lösungen sind
\begin{equation} 
p^{(0)\rightarrow(1)} =\ \sum_{jk} \psi_j^{(1)*}\psi_k^{(1)}\psi_j^{(0)}\psi_k^{(0)*}
\end{equation}

Die Bornschen Übergangswahrscheinlichkeiten zwischen allgemeinen Zuständen, also auch den nicht beobachtbaren, sind
\begin{equation} 
p^{(0)\rightarrow(1)} =\ \sum_{jklm} \psi_{jk}^{(1)*}\psi_{lm}^{(1)}\psi_{jk}^{(0)}\psi_{lm}^{(0)*}
\end{equation}

\subsection{Zeitentwicklung}
Oft wird die Zeitentwicklung durch einen parametrisch von der Zeitdifferenz $t_1-t_0$ abhängigen Zeitentwicklungsoperator ausgedrückt. Für die Übergangswahrscheinlichkeit von einem Ausgangszustand $\ket{\psi^{(0)}}$ zur Zeit $t_0$ zu einem Endzustand $\ket{\psi^{(1)}}$ zur Zeit $t_1$ gilt bei einem nicht explizit zeitabhängigen Operator $\hat{H}$
\begin{equation} 
\label{eq:time_evolution}
p^{(0)\rightarrow(1)} =\ \Bigl| \bra{\psi^{(1)}}\ket{\hat{U}(t_1-t_0)\,\psi^{(0)}} \Bigr|^2
\end{equation}
Dazu muss bemerkt werden:
\begin{itemize}
\item Ein zeitabhängiger Hamiltonoperator modelliert üblicherweise einen zeitabhängigen Einfluss der Umgebung. Betrachten wir das All als Quantenwelt, dann gibt es davon keine Umgebung. Dort stünde ein zeitabhängiger Operator $\hat{H}$ für zeitabhängige Naturgesetze. Diese Möglichkeit mag es geben, wir betrachten sie hier aber nicht.
\item Was genau sich in \eqref{eq:time_evolution} zeitlich entwickelt, ist ontologisch nicht klar. So lange die Wahrscheinlichkeit sich dabei nicht ändert, kann die Zeitentwicklung beliebig zwischen Operator und Vektoren hin- und hergeschoben werden, was in den gebräuchlichen "Bildern", also Schrödinger-, Heisenberg- und Wechselwirkungsbild, ausgenutzt wird.
\item $\ket{\psi^{(0)}}$ und $\ket{\psi^{(1)}}$ sind Vektoren aus dem $\mathscr{H}$ Unterraum. Statt abstrakter Vektoren aus dem $\mathscr{E}$ Unterraum treten in der Formel nur Zeiteigenwerte $t_0$, $t_1$ als Parameter auf. Auch in der relativistischen Quantenmechanik wird das \href{http://www.itp.uni-bremen.de/~noack/dirac.pdf}{invariante Skalarprodukt} definiert als \emph{drei}dimensionales Integral proportional zu 
$\int \frac{\mathrm{d}^3 \vec{p}}{p_0}\ \psi^*(p)\, \phi(p)$. Diese Integration "auf der Massenschale" entspricht der Vermeidung der Zustände \eqref{eq:general_state}, bei denen $j \neq k$ ist. $p_0$ gehört zwar zum $\mathscr{E}$ Raum, ist aber in dieser Formel nicht unabhängig sondern eine Funktion von $\vec{p}$ aus dem $\mathscr{H}$ Raum. Diese Ungleichbehandlungen wirken verstörend, wenn man sie mit der zugrunde liegenden symmetrischen Aufgabenstellung \eqref{eq:system} vergleicht.
\item Es handelt sich bei \eqref{eq:time_evolution} um die ursprüngliche Formulierung \eqref{eq:orig_Born} der Bornschen Wahrscheinlichkeit. Der Verschränkungsprozess mit der Umgebung ist nicht ausmodelliert.
\end{itemize}

In der Energiebasis des $\mathscr{H}$ Raumes ist $\hat{U}$ bei einem nicht explizit zeitabhängigen Hamilton-Operator diagonal. Die Matrixelemente in der Energiebasis sind 
\begin{equation} 
U_{E_j E_k} = \delta_{E_j E_k} e^{-\frac{\mathrm{i}}{\hbar}E_j(t_1-t_0)}
= 
\delta_{E_j E_k}(e^{\frac{\mathrm{i}}{\hbar}E_j t_1})^* e^{\frac{\mathrm{i}}{\hbar}E_j t_0}
\end{equation}
In Komponenten stellt sich \eqref{eq:time_evolution} dann so dar:
\begin{equation} 
p^{(0)\rightarrow(1)} =\ 
\Bigl| \sum_j
(\leftidx{^\mathscr{H}}{\psi_{E_j}}^{(1)} e^{\frac{\mathrm{i}}{\hbar}E_j t_1})^*\ 
\leftidx{^\mathscr{H}}{\psi_{E_j}}^{(0)} e^{\frac{\mathrm{i}}{\hbar}E_j t_0}
\Bigr|^2
\end{equation}
In den Exponentialfunktionen erkennen wir die Komponenten von Zeiteigenvektoren in der Energiedarstellung aus \eqref{eq:time_components} wieder.
Wir haben es also eigentlich mit Skalarprodukten von Vektoren aus dem $\mathscr{H}\otimes\mathscr{E}$ Raum zu tun
\begin{equation*}
\ket{\psi^{(0)}}\ =\ \sum_j \leftidx{^\mathscr{H}}{\psi_{E_j}}^{(0)}\ ^\mathscr{H}\ket{E_j} 
\ \otimes\ ^\mathscr{E}\psi_{t_0 E_j} \leftidx{^\mathscr{E}}{\ket{E_j}}
\ =\ \sum_j \leftidx{^\mathscr{H}}{\bra{E_j}\ket{\psi_{E_j}}} ^\mathscr{E}\bra{E_j}\ket{t_0} \ ^\mathscr{H}\ket{E_j}\ \otimes\ ^\mathscr{E}\ket{E_j}
\end{equation*}
\begin{equation*}
\ket{\psi^{(1)}}\ =\ \sum_j \leftidx{^\mathscr{H}}{\psi_{E_j}}^{(1)}\ ^\mathscr{H}\ket{E_j} 
\otimes\ ^\mathscr{E}\psi_{t_1 E_j} \leftidx{^\mathscr{E}}{\ket{E_j}}
\ =\ \sum_j \leftidx{^\mathscr{H}}{\bra{E_j}\ket{\psi_{E_j}}} ^\mathscr{E}\bra{E_j}\ket{t_1} \ ^\mathscr{H}\ket{E_j}\ \otimes\ ^\mathscr{E}\ket{E_j} 
\end{equation*}
also mit Vektoren der Form \eqref{eq:general_solution}.

%QM handelt also nur von Wahrscheinlichkeiten für einen Unterraum.
%Kann nicht richtig sein, da wegen spez. Relat. sich das Weltgeschehen für einen relativ bewegeten Beobachter in einem anderen Unterraum von $\mathrm{H}\otimes\mathrm{E}$ abspielt.

%Sollen die QM beider Beobachter gültige Theorien sein, dann muss sich für einen Beobachter seine QM in seinem Unterraum abspielen, während sich für den anderen Beobachter seine QM in einem anderen Unterraum abspielt. 

%Sollen beide Beobachter dasselbe Weltgeschehen beobachten, dann muss es einen Filter geben, so dass jeder Beobachter nur seinen Unterraum des beobachten kann. Der Beobachter ist dieser Filter.

\section{Eine Abfolge von Blockuniversen}

Wir wollen nun die deterministische Zeitentwicklung durch einen stochastischen Prozess ersetzen. Der Ausgangspunkt dazu ist ein Prozessschritt wie in \eqref{eq:collapse}. Das heißt: ein Beobachter nimmt einen "Kollaps" wahr, bei dem ein zwischen $\mathscr{H}$, $\mathscr{E}$ und der Umgebung $\mathrm{E}$ verschränkter Zustand entschränkt wird. Dieser Prozessschritt soll sein
\begin{equation*} 
\sum_{jk} \psi_{jk}\ \ket{e_{jk}} \otimes \ket{\lambda_j} \otimes \ket{t_k} 
\ \xrightarrow{p_{jk}\ =\ \left|\psi_{jk}\right|^2}\ 
\ket{e_{jk}} \otimes \ket{\lambda_j} \otimes \ket{t_k}
\end{equation*}
Es werden also beobachtet der Zustand $\ket{\lambda_j}$ \emph{und gleichzeitig} die Zeit $\ket{t_k}$, \emph{nicht} der Zustand $\ket{\lambda_j}$ \emph{zu} einer Zeit $t$. Dadurch ist die Dynamik am Ende angekommen. Jede weitere Beobachtung liefert immer wieder dieselben Eigenwerte und damit auch dieselbe Zeit $t_k$. 

Die Frage stellt sich nun, zu welcher Zeit die Beobachtung der Zeit eigentlich stattfindet. Diese Zeit, in der alles Wahrgenommene beobachtet wird, kann natürlich nicht mehr dieselbe Zeit sein wie die, die sich erst dann preisgibt, wenn sie durch eine bestimmte Entschränkung beobachtet wird. Da wir außer dem Beobachter sonst nichts mehr haben, muss diese außerhalb des physikalischen Modells befindliche Zeit zum Beobachter gehören. Es muss eine psychische Zeit sein.

Für jeden physikalischen Zeitpunkt $t_k$ müssen die Wahrscheinlichkeiten denen von \eqref{eq:collapse} entsprechen. Die $\left|\psi_{jk}\right|^2$ müssen also proportional sein zu den $\left|\psi_{\lambda_j}\right|^2$, das heißt wir können setzen
\begin{equation*}
\psi_{jk} =  \psi_{\lambda_j} \psi_{t_k}
\end{equation*}
Die $\psi_{t_k}$ sind frei wählbare komplexe Zahlen, es müssen nur die Normierungsbedingungen
\begin{equation*}
\sum_j \left|\psi_{\lambda_j}\right|^2 = 1 \quad\quad
\sum_{k} \left|\psi_{t_k}\right|^2 = 1
\end{equation*}
erfüllt sein. Damit haben wir 
\begin{equation} 
\label{eq:collapse_with_time}
\sum_{jk} \psi_{\lambda_j}\psi_{t_k} \ \ket{e_{jk}} \otimes \ket{\lambda_j} \otimes \ket{t_k} 
\ \xrightarrow{p_{jk}\ =\ \left|\psi_{\lambda_j}\right|^2\ \cdot\ \left|\psi_{t_k}\right|^2}\ 
\ket{e_{jk}} \otimes \ket{\lambda_j} \otimes \ket{t_k}
\end{equation}

Den kontinuierlich parametrisierten Operator $\hat{U}(t_1-t_0)$ nähern wir an durch Potenzen $\hat{u}^n$ eines kleinstmöglichen Zeitentwicklungsoperators $\hat{u}$. Dabei stellen wir uns $t_1-t_0$ vor als Vielfaches einer kleinstmöglichen Zeitdifferenz $t_p$.
\begin{equation*}
t_1-t_0\ =\ n t_p \quad\quad 
\hat{u} = \hat{U}(t_p) \quad\quad 
\hat{u}^n = \hat{U}(t_1-t_0)
\end{equation*}
Der Operator $\hat{U}$ ist dabei der aus \eqref{eq:decoherence} und nicht der aus \eqref{eq:time_evolution}. Er wirkt im Produktraum des Quantendings und seiner Umgebung, aber nicht im Unterraum $\mathscr{E}$, welcher von den Zeiteigenvektoren aufgespannt wird.

Die deterministische Zeitabhängigkeit des Ausgangszustands im konventionellen Schrödingerbild drückt sich aus durch
\begin{equation*}
\ket{\psi(t)}\ =\ \hat{U}(t-t_0)\ket{\psi^{(0)}}
\end{equation*} 
In unserem neuen Bild transportiert die k-malige Anwendung von $\hat{u}$ vom Ausgangszustand zu jenem, der mit $\ket{t_k}$ verschränkt sein soll.
\begin{equation*}
\ket{\psi^{(k)}}\ =\ \hat{u}^k \ket{\psi^{(0)}}
\end{equation*} 

Damit eine deterministische Zeitentwicklung vorgetäuscht werden kann, muss die Überlagerung auf der linken Seite von \eqref{eq:collapse_with_time} damit diese Gestalt annehmen
\begin{equation}
\label{eq:final_superposition}
\sum_{jk} \psi_{\lambda_j}^{(0)}\psi_{t_k}\ \hat{u}^{k} \left(\,\ket{e^{(0)}} \otimes \ket{\lambda_j} \,\right) \otimes \ket{t_k}
\ \xrightarrow{}
\end{equation}
Diese kollabieren in der Beobachtung zu den Vektoren auf der rechten Seite von  \eqref{eq:collapse_with_time}. Die $|\psi_{\lambda_j}|^2$ haben wir durch \eqref{eq:final_superposition} so gewählt, dass sie die konventionelle Quantenmechanik wiedergeben, und die $\psi_{t_k}$ sind nach wie vor frei. Damit ein Kollaps stattfinden kann bzw. vor der Beobachtung eine Verschränkung vorliegt, müssen wenigstens 2 der $\psi_{t_k}$ ungleich 0 sein. 

Damit sind wir am Ziel angekommen. Allerdings besteht noch die Unschönheit, dass der Prozess nach der ersten Beobachtung zum Erliegen kommt. Wir brauchen einen weiteren Prozessschritt, der wieder eine Überlagerung der Form \eqref{eq:final_superposition} aufbaut. 

Dieser andere Prozessschritt ist für menschliche Beobachter verborgen, da wir postuliert haben, dass der Kollaps der Überlagerung das ist, das erfahren wird. 

Dieser andere Prozessschritt darf aus mehreren Unterschritten bestehen und kann aus dem Unterraum der Lösungsvektoren \eqref{eq:general_solution} hinausführen. Der Prozess muss nicht markovsch sein und darf ein Gedächtnis haben. All das können wir nicht wissen, wenn es nicht gelingt, weitere Beobachtungsperspektiven zu erschließen. 

\subsection{Gerichtete Zeit}

Die stochastischen Prozessschritte \eqref{eq:collapse_with_time} werden beobachtet und wechseln sich ab mit nicht beobachtbaren Prozessschritten. Das ergibt insgesamt eine Folge von Beobachtungen, die jeweils einen Zeitpunkt $t_k$ liefern. Der verborgene Prozess könnte theortisch Überlagerungen erzeugen, die uns ungewohnt abenteuerliche Folgen von Zeitpunkten $t_k$ liefern. Das uns erscheinende Fließen der Zeit im physikalischen Kanal kann dadurch dargestellt werden, dass ausgehend von $t_k$ nur Überlagerungen aus Zeiten $> t_k$ durch den verborgenen Prozess präpariert werden.

\begin{equation} 
\label{eq:preparation_with_time}
\ket{e^{(0)}} \otimes \ket{\lambda_j} \otimes \ket{t_k}
\ \xrightarrow{p_{n}}\ 
\sum_{l,m=k+1} \psi_{n,\lambda_l}\psi_{n,t_m} \ \ket{e_{lm}} \otimes \ket{\lambda_l} \otimes \ket{t_m} 
\end{equation}
Durch $p_n$ haben wir angedeutet, dass der verborgene Prozess mit verschiedenen  Wahrscheinlichkeiten in $n$ verschiedene Überlagerungen führen kann, ohne dass der Beobachter Mittel hätte, deren Unterschiedlichkeit festzustellen.

Im einfachsten Fall kann ein Fließen durch sichere ($n=1$) verborgene Prozessschritte jeweils auf eine Überlagerung aus 2 benachbarten Zeiten so erzeugt werden:

\begin{equation*} 
\ket{e^{(0)}} \otimes \ket{\lambda_j} \otimes \ket{t_k}
\ \xrightarrow{100\,\%}\ 
\sum_{l,m=k+1}^{m=k+2} \psi_{\lambda_l}\psi_{t_m} \ \ket{e_{lm}} \otimes \ket{\lambda_l} \otimes \ket{t_m} 
\end{equation*}

Ein einfaches Beispiel in einem Hilbertraum aus 2 Qutrits und unistochastichen, markovschen Prozessschritten enthält der Artikel \href{http://vermaschung.de/index.php?title=Warum_Panpsychismus%3F}{Über psychische und physikalische Zeit}. 

Hier soll ausdrücklich betont werden, dass die Zeiteigenvektoren $\{\ \ket{t_k}\ \}$ des Raums $\mathscr{E}$ zunächst willkürlich nummeriert werden können. Erst durch das Vorhandenensein eines entsprechenden Prozesses und eine bestimmte Beobachtungsperspektive wird eine Reihenfolge nahegelegt, die dadurch immer einen subjektiven Charakter hat.

\subsection{Und der Quanten-Zeno-Effekt?}

Natürlich stellt sich jetzt die Frage, wie der Quanten-Zeno-Effekt in die neue Dynamik passt. In der konventionellen Quantenmechanik soll er ja entstehen durch die Störung der Dynamik \eqref{eq:time_evolution} durch wiederholte Beobachtung. Nun haben wir aber \eqref{eq:time_evolution} nicht mehr und statt dessen nur noch einen beobachtbaren Typ von Prozessschritt. \eqref{eq:final_superposition} liefert uns immer die richtigen, zu den beobachteten Zeitpunkten passenden Amplituden, wodurch wir eigentlich mit dem Quanten-Zeno-Effekt kein Problem haben. Was aber macht den Unterschied zwischen "häufig" und "selten" beobachten aus?

\eqref{eq:final_superposition} liefert Zeiten und stellt damit für den Beobachter eine Uhr dar. Nun will er diese für sein Quanten-Zeno-Experiment mehr oder weniger oft ablesen. Um dieses "mehr oder weniger oft" physikalisch abzusichern, muss er eine weitere "Laboruhr" haben, die er oft genug abliest. Diese Laboruhr bringt weitere Räume $\mathscr{H}$ und $\mathscr{E}$ in's Spiel. Der verborgene Prozess muss auch die Laboruhr geeignet präparieren, damit dem Beobachter auch durch sie ein Fluss der physikalischen Zeit erscheint.

Damit haben wir ein Quantending Q, dessen Zeiten wir beobachten
\begin{equation*}
\leftidx{^Q}t_1\ \rightarrow\ \leftidx{^Q}t_2\ \rightarrow \dots
\end{equation*}
und eine Laboruhr L\footnote{Am Ende ist sie natürlich auch ein Quantending. In der relativistischen Quantenfeldtheorie bringt jedes "Teilchen" seine eigene Zeitkoordinate mit. Oder anders ausgedrückt vergrößert ein Erzeugungsoperator die Fockraumbelegung um einen $\mathscr{H} \otimes \mathscr{E}$ Raum.}, die uns ebenfalls Zeiten liefert
\begin{equation*}
\leftidx{^L}t_1\ \rightarrow\ \leftidx{^L}t_2 \ \rightarrow \dots
\end{equation*}
Nun ist klar, was mit "häufig" und "selten" gemeint ist. Der Beobachter bekommt von der Laboruhr immer wieder Überlagerungen der Gestalt \eqref{eq:final_superposition} angeboten und liest daraus die Laborzeiten ab. Für eine Verschränkung des Quantensystems mit der Umgebung sorgt er aber nicht bei jeder Ablesung der Laboruhr, sondern mal bei enger, mal bei weiter außeinander liegenden Ablesungen. 

Das Quantensystem liefert ihm eine weitere Zeit. Selbst wenn sie gerichtet ist wie die Zeit der Laboruhr, könnte sie gegenüber dieser noch wilde Sperenzchen treiben. Dies widerspricht jedoch der empirischen Erfahrung, dass sich Uhren gut synchronisieren lassen. Damit sich Uhren synchronisieren lassen, muss der verborgene Prozess Zeiteigenvektoren aller Räume $\mathscr{E}$ geeignet verschränken. Anders ausgedrückt: Unterräume, die vom verborgenen Prozess in bestimmter Weise verschränkt werden, erscheinen uns als objektiv vergehende physikalische Zeit.

Wir tasten uns an die neue Dynamik heran und lassen zunächst die Dekohärenz durch Verschränkung mit der Umgebung außen vor wohl wissend, dass es immer möglich sein muss, einen Dekoärenzprozess zu finden, der uns dieselben Wahrscheinlichkeiten liefert wie die ursprüngliche Sicht. Wie in \eqref{eq:orig_Born} soll es "jederzeit" möglich sein, durch eine projektive Messung entweder nur die Laboruhr abzulesen oder zusätzlich noch das Quantensystem zu messen. Die Überlagerung \eqref{eq:final_superposition} erweitert um die Laboruhr aber ohne Berücksichtigung der Umgebung soll uns der verborgene Prozess in dieser Form herstellen:
\begin{equation}
\label{eq:superposition_with_lab}
\sum_{jkl}\ \psi_{t_k} \leftidx{^\mathrm{Q}}\psi_{\lambda_j}^{(0)}\ 
\leftidx{^\mathrm{Q}}{\hat{u}}^k 
\ket{\leftidx{^\mathrm{Q}}\lambda_j} 
\ \otimes\ \ket{\leftidx{^\mathrm{Q}}t_k}
\ \otimes\ \leftidx{^\mathrm{L}}{\hat{u}}^k \ket{ \leftidx{^\mathrm{L}}\lambda_l} 
\ \otimes\ \ket{\leftidx{^\mathrm{L}}t_k}
\ \xrightarrow{}
\end{equation}
Hierbei haben wir Zeiteigenvektoren mit exakt denselben Indizes paarweise verschränkt. So lange noch nichts anderes gemessen wurde, könnten wir durchaus zulassen, dass z.B. auch Terme mit $\ket{\leftidx{^\mathrm{Q}}t_k} \otimes \ket{\leftidx{^\mathrm{L}}t_{k+1}}$ zur Überlagerung beitragen. Dies würde bedeuten, dass Uhren in der Natur nicht exakt synchron laufen können. Wir beschränken uns nun aber auf den einfachen Fall exakter Synchronizität.

Von \eqref{eq:superposition_with_lab} aus können wir eine vollständige projektive Messung durchführen
\begin{equation}
\xrightarrow{p_{jk}}\ \leftidx{^\mathrm{Q}}\psi_{\lambda_j}^{(0)}\ 
\leftidx{^\mathrm{Q}}{\hat{u}}^k
\ket{\leftidx{^\mathrm{Q}}\lambda_j} 
\ \otimes\ \ket{\leftidx{^\mathrm{Q}}t_k}
\ \otimes\ \leftidx{^\mathrm{L}}{\hat{u}}^k \ket{ \leftidx{^\mathrm{L}}\lambda_l} 
\ \otimes\ \ket{\leftidx{^\mathrm{L}}t_k}
\end{equation}
Von hier aus stellt uns der verborgene Prozess wieder die Überlagerung \eqref{eq:superposition_with_lab} zusammen. Da wir in der $\lambda$-Basis gemessen haben, vereinfacht sich \eqref{eq:superposition_with_lab} speziell in dieser Basis zu 
\begin{equation}
\label{eq:superposition_with_lab_after_meas}
\sum_{kl}\ \leftidx{^\mathrm{Q}}{\hat{u}}^k
\ket{\leftidx{^\mathrm{Q}}\lambda_j} 
\ \otimes\ \ket{\leftidx{^\mathrm{Q}}t_k}
\ \otimes\ \leftidx{^\mathrm{L}}{\hat{u}}^k \ket{ \leftidx{^\mathrm{L}}\lambda_l} 
\ \otimes\ \ket{\leftidx{^\mathrm{L}}t_k}
\ \xrightarrow{}
\end{equation}
Wollen wir das nächste Mal in einer anderen Basis messen, so erscheint \eqref{eq:superposition_with_lab_after_meas} in dieser anderen Basis in der Gestalt \eqref{eq:superposition_with_lab}. Da wir beim Ablesen der Laborzeit die Messbasis immer beibehalten, bleibt es für L dagegen dauerhaft bei der einfacheren Form.

Oder wir können eine teilprojektive Messung durchführen, wenn wir nur die Laborzeit ablesen aber das Quantending in Ruhe lassen
\begin{equation}
\xrightarrow{|\psi_{t_k}|^2}\ 
\leftidx{^\mathrm{L}}{\hat{u}}^k \ket{ \leftidx{^\mathrm{L}}\lambda_l} 
\ \otimes\ \ket{\leftidx{^\mathrm{Q}}t_k}
\ \otimes\ \ket{\leftidx{^\mathrm{L}}t_k}
\ \otimes\ \sum_{j}\ \leftidx{^\mathrm{Q}}\psi_{\lambda_j}^{(0)}\ 
\leftidx{^\mathrm{Q}}{\hat{u}}^k
\ket{\leftidx{^\mathrm{Q}}\lambda_j} 
\end{equation}
Von hier aus stellt der verborgene Prozess wieder \eqref{eq:superposition_with_lab} her, in der nicht-vereinfachten Form, da die Überlagerung in Q noch vorhanden ist.

Damit kann die neue Dynamik die Wahrscheinlichkeiten für Ereignisse inklusive Ausbremsen durch häufige Beobachtungen liefern, allerdings mit einem wichtigen Unterschied. Die Summe in \eqref{eq:preparation_with_time} muss über mindestens 2 verschiedene Zeiten gehen, da nur eine Überlagerung erfahren werden kann. Selbst wenn dort statt der Summe ein Integral stünde, müsste das Integral über ein endliches Intervall laufen. Der Erwartungswert der physikalischen Zeit kann deswegen nicht gleich der letzten gemessenen Zeit sein. Im Unterschied zur Formulierung des Quanten-Zeno-Effekts mit kontinuierlicher deterministischer Zeitentwicklung ist die Voraussage der neuen Dynamik: 

\emph{Es ist nicht möglich, die Bewegung komplett anzuhalten.} Denn die Beobachtbarkeit setzt eine Veränderung des Zustands seit der letzten Beobachtung voraus. Vielmehr erwarten wir eine Mindestveränderung der physikalischen Zeit mit jeder Beobachtung, die durch die Präparation, die der verborgene Prozess uns anbietet, bestimmt ist\footnote{also eine Art Planck-Zeit}.

\textbf{Im Dekohärenzbild} 

In der Dekohärenztheorie ergibt sich ein anderes Bild. Es ist nicht so, dass ein Zustand \eqref{eq:superposition_with_lab} stets die Wahl erlaubt, die Laboruhr abzulesen oder ein Quantending zu messen oder beides. Vielmehr muss die Entscheidung bereits davor gefallen sein. Entweder wird nur die Laboruhr mit der Umgebung verschränkt, dann kann auch nur sie gemessen werden. Oder es wird zusätzlich das Quantending mir der Umgebung verschränkt. Die Entscheidung, worüber im nächsten Schritt Information erlangt werden kann, muss \emph{vor} dem verborgenen Prozess getroffen werden.

Um über die physikalische Zeit informiert zu sein, muss der Experimentator dafür sorgen, dass sich seine Laboruhr L stets mit der Umgebung E verschränkt. Der verborgene Prozess soll ihm solche Überlagerungen anbieten
\begin{equation}
\label{eq:partial_superposition_with_lab}
\sum_{jk} \ \psi_{t_k} \leftidx{^\mathrm{Q}}\psi_{\lambda_j}^{(0)}\ 
\leftidx{^\mathrm{E \otimes L}}{\hat{u}}^k \left( \ket{e^{(0)}} \otimes \leftidx{^\mathrm{L}}{\ket{\lambda_l}} \right) 
\otimes \leftidx{^\mathrm{Q}}{\hat{u}}^{k}\, \leftidx{^\mathrm{Q}}{\ket{\lambda_j}}
\otimes \leftidx{^\mathrm{Q}}{\ket{t_k}}
\otimes \leftidx{^\mathrm{L}}{\ket{t_k}}
\ \xrightarrow{}
\end{equation}
Der linke Operator $\hat{u}$ wirkt auf Laboruhr und Umgebung, der rechte auf das Quantending. Insgesamt haben wir einen Operator
\begin{equation}
\label{eq:unitary_2_parts}
\hat{u}\ =\ \leftidx{^\mathrm{E \otimes L}}{\hat{u}}\ \otimes\ \leftidx{^\mathrm{Q}}{\hat{u}}
\end{equation}
Von \eqref{eq:partial_superposition_with_lab} aus kann er eine physikalische Zeit ablesen
\begin{equation*}
\xrightarrow{} \ \ket{e_{lk}} 
\otimes \leftidx{^\mathrm{L}}{\ket{\lambda_l}} \,
\otimes \leftidx{^\mathrm{Q}}{\ket{t_k}}
\otimes \leftidx{^\mathrm{L}}{\ket{t_k}}
\otimes \sum_{j} \, \leftidx{^\mathrm{Q}}{\psi_{\lambda_j}}^{(0)}\ 
\leftidx{^\mathrm{Q}}{\hat{u}}^{k}\ 
\leftidx{^\mathrm{Q}}{\ket{\lambda_j}}
\end{equation*}

Will der Experimentator dagegen Informationen über das Quantending erhalten, dann muss er es auch mit der Umgebung verschränken. Dazu muss der verborgene Prozess einen anderen Operator $\hat{u}$ synthetisieren.
\begin{equation}
\label{eq:full_superposition_with_lab}
\sum_{jk} \ \psi_{t_k}\ \leftidx{^\mathrm{Q}}{\psi_{\lambda_j}}^{(0)}\ 
\hat{u}^k \left( \ket{e^{(0)}} \otimes \leftidx{^\mathrm{L}}{\ket{\lambda_l}} 
\otimes \leftidx{^\mathrm{Q}}{\ket{\lambda_j}} \right) 
\otimes \leftidx{^\mathrm{Q}}{\ket{t_k}}
\otimes \leftidx{^\mathrm{L}}{\ket{t_k}}
\ \xrightarrow{}
\end{equation}
Von hier aus kann er die volle Information erlangen.
\begin{equation*}
\xrightarrow{}\ \ket{e_{jl}} 
\otimes \leftidx{^\mathrm{L}}{\ket{\lambda_l}} 
\otimes \leftidx{^\mathrm{Q}}{\ket{\lambda_j}}
\otimes \leftidx{^\mathrm{L}}{\ket{t_k}}
\otimes \leftidx{^\mathrm{Q}}{\ket{t_k}}
\end{equation*}
Nun stellt sich natürlich die Frage, warum und aufgrund wovon der verborgene Prozess häufig \eqref{eq:partial_superposition_with_lab} und seltener \eqref{eq:full_superposition_with_lab} präparieren sollte. Dies entspräche einer häufigen Ablesung der Laboruhr mit ab und zu stattfindenden Quantenmessungen. Einen unitären Operator \eqref{eq:unitary_2_parts}, der nur in bestimmten Potenzen\footnote{und damit zu bestimmten physikalischen Zeiten} $k$ eine Verschränkung des Quantendings mit der Umgebung durchführt, gibt es nicht.

Soll der verborgene Prozess Teil einer objektiven Wirklichkeit sein, so ist die Erwartung eher, dass er von einem bestimmten Ausgangszustanda ausgehend immer das Gleiche präpariert, wobei allerdings die $\psi_{t_k}$ frei bleiben können. Und der Experimentator kann der Meinung sein, dass er aufgrund seiner "freien Willensentscheidung" einen Schalter umlegt und damit entscheidet, ob \eqref{eq:partial_superposition_with_lab} oder \eqref{eq:full_superposition_with_lab} durchgeführt wird, also die "kleine" oder die "große" Verschränkung entsteht.

Um dies zu modellieren wird im einfachsten Fall ein weiteres Qubit-Paar benötigt. Zunächst stellen wir fest, dass der Übergang vom Ausgangszustand aus hin zu   \eqref{eq:partial_superposition_with_lab} eine unitäre Operation ist, die wir als $\hat{U_0}$ bezeichnen wollen.
\begin{equation}
\eqref{eq:partial_superposition_with_lab} = \hat{U}_0
\left(
\ket{e^{(0)}} 
\otimes \leftidx{^\mathrm{L}}{\ket{\lambda_l}} 
\otimes \leftidx{^\mathrm{Q}}{\ket{\psi^{(0)}}}
\otimes \leftidx{^\mathrm{Q}}{\ket{t_0}}
\otimes \leftidx{^\mathrm{L}}{\ket{t_0}}
\right)
\end{equation} 
Genauso ist der Übergang vom Ausgangszustand aus hin zu \eqref{eq:full_superposition_with_lab} eine unitäre Operation, die wir als $\hat{U_1}$ bezeichnen wollen.
\begin{equation}
\eqref{eq:full_superposition_with_lab} = \hat{U}_1
\left(
\ket{e^{(0)}} 
\otimes \leftidx{^\mathrm{L}}{\ket{\lambda_l}} 
\otimes \leftidx{^\mathrm{Q}}{\ket{\psi^{(0)}}}
\otimes \leftidx{^\mathrm{Q}}{\ket{t_0}}
\otimes \leftidx{^\mathrm{L}}{\ket{t_0}}
\right)
\end{equation} 
Das Qubit-Paar soll aus einem $\mathscr{H}$-artigen Bit und einem $\mathscr{E}$-artigen Zeitbit bestehen. $\leftidx{^\mathscr{H}}{\ket{0}}$ soll dafür stehen, dass eine Messung an Q durchgeführt werden soll. $\leftidx{^\mathscr{H}}{\ket{1}}$ soll dagegen bedeuten, dass im nächsten Schritt nur die Laborzeit abgelesen werden soll. $\leftidx{^\mathscr{E}}{\ket{0}}$ und $\leftidx{^\mathscr{E}}{\ket{1}}$ sind irgendwelche verschiedenen Zustände im Qubit-Zeitraum. Wir definieren uns die Qubit-Produktzustände
\begin{equation*}
\ket{00} \equiv \leftidx{^\mathscr{H}}{\ket{0}}\ \otimes\ \leftidx{^\mathscr{E}}{\ket{0}}
\quad\quad\quad\quad
\ket{11} \equiv \leftidx{^\mathscr{H}}{\ket{1}}\ \otimes\ \leftidx{^\mathscr{E}}{\ket{1}}
\end{equation*}
\begin{equation*}
\ket{++} \equiv \frac{1}{\sqrt(2)}\left(\ket{00}+\ket{11}\right)
\quad\quad\quad\quad
\ket{--} \equiv \frac{1}{\sqrt(2)}\left(\ket{00}-\ket{11}\right)
\end{equation*}
$\ket{++}$ und $\ket{--}$ verschränken also $\mathscr{H}$ und $\mathscr{E}$.

Als nächstes erweitern wir den Ausgangszustand um das Qubit-Paar. Er soll sein 
\begin{equation*}
\ket{e^{(0)}} 
\otimes \leftidx{^\mathrm{L}}{\ket{\lambda_l}} 
\otimes \leftidx{^\mathrm{Q}}{\ket{\psi^{(0)}}}
\otimes \leftidx{^\mathrm{Q}}{\ket{t_0}}
\otimes \leftidx{^\mathrm{L}}{\ket{t_0}}
\otimes \ket{00}
\end{equation*}
wenn als nächstes nur die Laborzeit abgelesen werden soll und 
\begin{equation*}
\ket{e^{(0)}} 
\otimes \leftidx{^\mathrm{L}}{\ket{\lambda_l}} 
\otimes \leftidx{^\mathrm{Q}}{\ket{\psi^{(0)}}}
\otimes \leftidx{^\mathrm{Q}}{\ket{t_0}}
\otimes \leftidx{^\mathrm{L}}{\ket{t_0}}
\otimes \ket{11}
\end{equation*}
wenn eine vollständige Messung durchgeführt werden soll. Nun können wir leicht einen unitären Operator $\hat{U}$ angeben, der in Abhängigkeit des Ausgangszustands zu \eqref{eq:partial_superposition_with_lab} oder \eqref{eq:full_superposition_with_lab} führt.
\begin{equation*}
\hat{U}\ =\ \hat{U}_0 \otimes \ket{++}\bra{00}\ +\ \hat{U}_1 \otimes \ket{--}\bra{11}
\end{equation*}
Von $\ket{00}$ aus führt uns dieser Operator zu 
\begin{equation*}
\eqref{eq:partial_superposition_with_lab} \otimes \ket{++}
\end{equation*}
und von $\ket{11}$ zu 
\begin{equation*}
\eqref{eq:full_superposition_with_lab} \otimes \ket{--}
\end{equation*}
Bei der Beobachtung zerfällt nun die Verschränkung \eqref{eq:partial_superposition_with_lab} bzw. \eqref{eq:full_superposition_with_lab} zugleich mit der Verschränkung des Qubit-Paares und damit fällt die Entscheidung, ob als nächstes nur die Laborzeit abgelesen werden soll oder eine vollständige Messung durchgeführt werden soll. 

\textbf{Ein Beispiel für $\hat{u}$ in \eqref{eq:full_superposition_with_lab}}

Ein zur Vorbereitung einer Messung geeigneter Operator $\hat{U}$ in \eqref{eq:decoherence} ist
\begin{equation*}
\hat{U}(t-t_0)\ =\ \mathrm{exp}\left(-\frac{i(t-t_0)}{\hbar}f(\hat{\lambda})\otimes \leftidx{^E}{\hat{H}}\right)
\end{equation*}
mit irgendeiner operatorwertigen Funktion $f$ und einem hermiteschen Operator $\leftidx{^E}{\hat{H}}$, der nur auf die Umgebung wirkt. $\lambda_j$ sind die Eigenwerte\footnote{In diesem Beispiel sind sie ohne physikalische Einheit.} von $\hat{\lambda}$.
Angewendet auf den Zustand von \eqref{eq:decoherence} ergibt sich
\begin{equation*}
\hat{U}(t-t_0)\left(\sum_j \psi^{(0)}_{\lambda_j} \ket{\lambda_j}\otimes\ket{e^{(0)}}\right)
\ =\ \sum_j \psi^{(0)}_{\lambda_j} \ket{\lambda_j} 
\otimes \mathrm{exp}\left(-\frac{i(t-t_0)}{\hbar}f(\lambda_j)\leftidx{^E}{\hat{H}}\right)
\ket{e^{(0)}} 
\end{equation*}
\begin{equation*}
\ \equiv\ 
\sum_j \psi^{(0)}_{\lambda_j} \ket{\lambda_j} 
\otimes \ket{e_j(t-t_0)} 
\end{equation*}
wodurch $\hat{U}$ das Quantending mit der Umgebung verschränkt\footnote{Die Exponentialfunktion verschränkt Operatorpotenzen: $\mathrm{exp}(\hat{a}\otimes\hat{b})=\sum \frac{1}{n\mathrm{!}}\ \hat{a}^n\otimes\hat{b}^n$} hat unter Beibehaltung der Amplituden $\psi^{(0)}$ und damit der Bornschen Wahrscheinlichkeiten von \eqref{eq:collapse}.

Der Bezug zur neuen Dynamik geht über $t_p$ und ist für \eqref{eq:full_superposition_with_lab} einfach
\begin{equation*}
\hat{u} = \mathrm{exp}\left(
-\frac{\mathrm{i}t_p}{\hbar}\ f\left(\leftidx{^L}{\hat{\lambda}}\right)\ \otimes\  g\left(\leftidx{^Q}{\hat{\lambda}}\right)\ \otimes\ \leftidx{^E}{\hat{H}}
\right)
\end{equation*}

-----------------
diskret

Schrödi + 1 Kollaps
\begin{equation} 
p^{(0)\rightarrow(1)} =\ \Bigl| \sum_{x'x} \psi^{(1)*}_{x'}\ U(\Delta t)_{x'x}\ \psi^{(0)}_x \Bigr|^2
\end{equation}
Elementares $\Delta t$ Planck-Zeit $\hat{U} = \hat{u}^{t/t_{Planck}}$
\begin{equation} 
p^{(0)\rightarrow(1)} =\ \Bigl| \sum_{x'x} \psi^{(1)*}_{x'}\ u_{x'x}\ \psi^{(0)}_x \Bigr|^2
\end{equation}
1 Zwischenbasis, 2 Kollapse
\begin{equation} 
\ket{\psi}^{(0)} =  \sum_{x} \psi^{(0)}_x \ket{x} \otimes \ket{t_0} \quad\quad
\ket{\psi}^{(1)} =  \sum_{x} \psi^{(1)}_x \ket{x} \otimes \ket{t_1}
\end{equation}
Zwischenbasis $\{\ket{z}\}$ $N \cdot M$ Elemente
\begin{equation} 
\begin{matrix}
\ket{x} \otimes \ket{t} = \sum_z V_{xt\, z}\ \ket{z} && 
\ket{z} = \sum_{xt} V^*_{z\, xt}\ \ket{x} \otimes \ket{t} 
\\ \\
\bra{x} \otimes \bra{t} = \sum_z V^*_{z\, xt}\ \bra{z} &&
\bra{z} = \sum_{xt} V_{xt\, z}\ \bra{x} \otimes \bra{t} 
\end{matrix}
\end{equation}
\begin{equation} 
p^{(0)\rightarrow(z)} =\ \Bigl| \bra{z} \hat{V} \left(\ket{x}\otimes\ket{t_0}\right)\Bigr|^2
= \Bigl|\sum_{xt\, x'} \psi^{(0)}_{x'} \bra{x'} \otimes \bra{t} V_{xt\, z}\ \ket{x} \otimes \ket{t_0}\Bigr|^2 
\end{equation}
\begin{equation} 
p^{(0)\rightarrow(z)} =\ \Bigl|\sum_{x} \psi^{(0)}_{x} V_{xt_0\, z}\ \Bigr|^2 
\end{equation}
analog
\begin{equation} 
p^{(z)\rightarrow(1)} =\ \Bigl|\sum_{x} \psi^{(1)}_{x} V_{xt_1\, z}\ \Bigr|^2 
\end{equation}
Gesamtübergangswahrscheinlichkeit
\begin{equation} 
p^{(0)\rightarrow(1)} =\ \sum_z\ p^{(0)\rightarrow(z)}\ p^{(z)\rightarrow(1)}
= \ \sum_{z}\ \Bigl(\ \Bigl|\sum_{x} \psi^{(1)}_{x} V_{xt_1\, z}\ \Bigr|^2\ \Bigl|\sum_{x} \psi^{(0)}_{x} V_{xt_0\, z}\ \Bigr|^2\ \Bigr)
\end{equation}
\begin{equation} 
p^{(0)\rightarrow(1)} 
= \ \sum_{z}\ \Bigl|\sum_{x'x} \psi^{(1)}_{x'} V_{x't_1\, z}\ V_{xt_0\, z}\ \psi^{(0)}_{x} \Bigr|^2  
\end{equation}

%\begin{equation} 
%p^{(0)\rightarrow(1)} 
%= \ \sum_{z}\ \Bigl|\sum_{x'x} \psi^{(1)}_{x'} v_{z\, x'}\ w_{z\, x}\ \psi^{(0)}_{x} %\Bigr|^2  
%\end{equation}
Also 
\begin{equation} 
\boxed{
\quad\Bigl| \sum_{x'x} \psi^{(1)*}_{x'}\ u_{x'x}\ \psi^{(0)}_x \Bigr|^2 
\ =\ \sum_{z}\ \Bigl|\sum_{x'x} \psi^{(1)}_{x'} V_{x't_1\, z}\, V_{xt_0\, z}\ \psi^{(0)}_{x} \Bigr|^2 \quad
}
\end{equation}

Ausmultipliziert links
\begin{equation} 
\sum_{x'''x''x'x}\psi^{(1)}_{x'''}\psi^{(0)*}_{x''}\psi^{(1)*}_{x'}\psi^{(0)}_x
\ u^*_{x'''x''}\ u_{x'x}
\end{equation}
rechts
\begin{equation} 
\sum_{z\,x'''x''x'x}\ 
\psi^{(1)}_{x'''}\psi^{(0)*}_{x''}\psi^{(1)*}_{x'}\psi^{(0)}_{x}
\, V_{x'''t_1\, z}\, V^*_{x''t_0\, z}\, V^*_{x't_1\, z}\, V_{xt_0\, z}
\end{equation}
Muss für alle $\ket{\psi^{(0)}}$ und $\ket{\psi^{(1)}}$ gelten, deshalb
\begin{equation} 
\boxed{
u^*_{x'''x''}\ u_{x'x} = \sum_z \, V_{x'''t_1\, z}\, V^*_{x''t_0\, z}\, V^*_{x't_1\, z}\, V_{xt_0\, z} 
}
\end{equation}
Unitarität von $u$ und $V$ sind zusätzlich zu fordern!
\begin{equation} 
\sum_{x''} u_{x'x''}\, u^*_{xx''} = \delta_{x'x} \quad\quad 
\sum_{z} V_{x't'\, z} V^*_{xt\, z} = \delta_{x'x}\ \delta_{t't} \quad\quad
\sum_{xt} V_{xt\, z'} V^*_{xt\, z} = \delta_{z'z}
\end{equation}

-----------------

\eqref{eq:time_evolution} diagnonalisiert durch unitäre Matrix $W$ im x-Unterraum. D.h. auf $V$ wirkt $W \otimes \mathrm{1}$.
\begin{equation}
U_{x'x\,t't} = e^\mathrm{i\varphi_x(t'-t)}\,\delta_{x'x}
\end{equation}
\begin{equation}
\begin{split}
\boxed{
e^\mathrm{i(\varphi_x-\varphi_{x''})(t'-t)}\,\delta_{x'''x''}\,\delta_{x'x}\ =\ \sum_z \, V_{x'''t'\, z}\, V^*_{x''t\, z}\, V^*_{x't'\, z}\, V_{xt\, z} 
} \\
\varphi_x \in \mathbb{R} \\ V_{xt\, z} \in \mathbb{C} \\
x \in \{1,2,...,N_x\}\\ t \in \{1,2,...,N_t\}\\ z \in \{1,2,...,N_x \cdot N_t\} 
\end{split}
\end{equation}

Nur noch Unitarität von $V$ ist zusätzlich zu fordern!
\begin{equation} 
\sum_{z} V_{x't'\, z} V^*_{xt\, z} = \delta_{x'x}\ \delta_{t't} \quad\quad
\sum_{xt} V_{xt\, z'} V^*_{xt\, z} = \delta_{z'z}
\end{equation}
testsetstestestestestreterte 
ert
ert
ert

\end{document}
