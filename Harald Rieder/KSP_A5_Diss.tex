%% Vorlage f�r Dissertationen

% Verbesserte Silbentrennung im deutschen nach dem aktuellen Stand
\RequirePackage[ngerman=ngerman-x-latest]{hyphsubst}
% Dokumentenklasse laden (keine Punkte hinter der Nummerierung, Schriftgr��e 10pt, Sprachpakete english und neue deutsche Rechtschreibung einbinden (deutsch ist Standard))
% auch hier Seitengr��e einstellen, indem a4paper durch a5paper oder umgekehrt ersetzt wird.
\documentclass[numbers=noenddot,a5paper,10pt,english,ngerman]{scrbook}
%\documentclass[halfparskip,numbers=noenddot,a5paper,10pt,english,ngerman]{scrbook}

% Weitere Pakete hier einf�gen
\usepackage{booktabs}
\usepackage{wrapfig}

\usepackage[all]{nowidow} %verhindert Hurenkinder
\usepackage{amsmath}
\usepackage{amsfonts}
\usepackage{amssymb}
\usepackage{graphicx}
\usepackage[]{acronym}
\usepackage{trfsigns}
\usepackage{cite}
\usepackage{caption} % for subfigures
\usepackage{subcaption} % for subfigures
\usepackage{enumitem}
%\usepackage{showframe} % Seitenraender anzeigen
%\usepackage{epstopdf}
%\usepackage{url}
%\usepackage{subfigure} % old package for subfigures
%\usepackage{geometry}
%\usepackage{wrapfig}
%\providecommand\phantomsection{}
%\usepackage{color}
%\usepackage{cancel}
%\usepackage{amsthm}
%\usepackage{ulem}
%\usepackage[utf8]{inputenc}
%\usepackage[automark]{scrpage2}
%\pagestyle{srcheadings}
%\usepackage{lmodern}
%\usepackage{empheq}
%\usepackage{xcolor}
\usepackage{calc}
%\usepackage{float}

%%%%%%%%%%%%%%%%%%%%%%%%%%%%%%%%%%%%%%%%%%%%%%%%%%

\hyphenation{
An-ten-ne
}

%%%%%%%%%%%%%%%%%%%%%%%%%%%%%%%%%%%%%%%%%%%%%%%%%%

\usepackage{multibib}
\newcites{journal}{Literatur/Eigene_Journal_Papers}
\newcites{conference}{Literatur/Eigene_Konferenz_Papers}
\usepackage{etoolbox}
\BeforeBeginEnvironment{thebibliography}{%Umdefinieren, um die eigenen Publikationen ohne Umbruch einzuf�gen
  \let\origchapter\chapter% save the original definition of \chapter
  \let\chapter\section%  make \chapter behave like \section
}
\AfterEndEnvironment{thebibliography}{%
  \let\chapter\origchapter% restore the original definition of \chapter
}

%%%%%%%%%%%%%%%%%%%%%%%%%%%%%%%%%%%%%%%%%%%%%%%%%%
% pdf-Titel
\newcommand{\pdftitle}{Titel\ der\ Dissertation}

% Autor
\newcommand{\autor}{Vorname\ Nachname}
% Option und Pakete einbinden
% -------------------- Einstellungen Seitenaufbau ------------------------------
% Seiteneinstellungen laden
\usepackage[nouppercase]{scrpage2}
% Keine Kopf/Fußzeile auf leeren Seiten
\usepackage{emptypage}
% Zeilenabstände einstellen auf standardmäßig 1,5fachen Zeilenabstand
\usepackage[singlespacing]{setspace}
% Seitenränder einstellen
\setlength{\topskip}{10.5pt} % Verhindern einer Fehlermeldung

% Seitengröße auf a4paper oder a5paper einstellbar (1. oder 2. usepackage wählen)
\usepackage[a5paper,headheight=1.5\baselineskip,top=25mm,lines=38,heightrounded=true,bindingoffset=12mm,textwidth=109mm,footnotesep=7mm]{geometry}
%\usepackage[a4paper,headheight=1.5\baselineskip,top=25mm,lines=46,heightrounded=true,bindingoffset=15mm,textwidth=160mm]{geometry}
% Zeilen auf der Seite verteilen ( Es wird kein Ausgleich des unteren Seitenrandes durch Dehnung der Absatzabstände durchgeführt.) 
\raggedbottom     
% Abstand zwischen Textkörper und Linie in der Kopfzeile
\setlength{\headsep}{8mm}
% Abstand zwischen Textkörper und Unterkante Fußzeile (Seitenzahlen)
\setlength{\footskip}{10mm}
% Verhindert  Erstzeileneinzüge
\setlength{\parindent}{0pt}
\setlength{\parskip}{2.5mm}
% Schusterjungen (einzelne Zeile unten auf der Seite) unterdrücken
\clubpenalty = 10000 
% Hurenkinder (einzelne Zeile oben auf der Seite) unterdrücken
\widowpenalty = 10000
\displaywidowpenalty = 10000
% Linie in der Kopfzeile definieren
\setheadsepline{0.5pt}
% Position und Größe der Fußnoten
\deffootnote[1.0em]{1.0em}{0em}{\textsuperscript{\thefootnotemark\ }}
% Mindestfüllgrad einer Seite mit einem Gleitobjekt
\renewcommand{\floatpagefraction}{0.7}
% Maximale Größe eines Gleitobjekts am unteren Seitenrand
\renewcommand{\topfraction}{0.8}
% Maximale Größe eines Gleitobjekts am oberen Seitenrand
\renewcommand{\bottomfraction}{0.8}
% mögliche Abstandsvergrößerung innerhalb einer Zeile bei unschönem Zeilenumbruch
\setlength{\emergencystretch}{4pt}
% Mindestanteil an Text auf einer Seite mit Gleitobjekt
\renewcommand{\textfraction}{0.1}
% Seitenbegrenzungen anzeigen, zum Überprüfen der Seitendarstellung
%\usepackage{showframe}

% -------------------- Einstellungen Zeichensatz ------------------------------
% Erscheinungsbild auf Deutsch umstellen
\usepackage{babel}
% Umlaute zulassen
\usepackage[T1]{fontenc}
% ß als Zeichen zulassen
\usepackage[latin1]{inputenc}
% Inhaltsverzeichnis richtig darstellen
\usepackage[titles]{tocloft}
% Auch bei den Kapiteln Punkte darstellen
\renewcommand{\cftchapdotsep}{\cftdotsep}
\renewcommand{\cftchapleader}{\cftdotfill{\cftchapdotsep}}
% Seitenzahlen bei Kapitel in Serifenloser Schriftart darstellen
\renewcommand{\cftchappagefont}{\fontfamily{phv}\normalsize\bfseries}
% Verzeichnisse aktualisieren

%Fonts im Inhaltsvz.
\renewcommand\cftchapfont{\fontfamily{phv}\normalsize\bfseries}
\renewcommand\cftsecfont{\fontfamily{phv}\fontsize{11}{11}}
%Fonts in Kapiteln und sections...
%\renewcommand\cftchappagefont{\fontfamily{phv}\normalsize\bfseries}
\renewcommand\cftsecpagefont{\fontfamily{phv}\fontsize{11}{11}}

\usepackage{makeidx}
% Adobe Times als Standardschriftart einstellen
\usepackage{newtxtext}
\usepackage{newtxmath}
% Schriftarten und -größen für die Überschriften vorgeben
%\addtokomafont{chapter}{\fontfamily{phv}\fontsize{17}{17}\bfseries}

%\addtokomafont{chapter}{\fontfamily{phv}\fontsize{17}{17}\bfseries}
%\addtokomafont{section}{\fontfamily{phv}\fontsize{13}{13}\bfseries}
%\addtokomafont{subsection}{\fontfamily{phv}\fontsize{11}{11}\bfseries}
%\addtokomafont{subsubsection}{\fontfamily{phv}\fontsize{9}{9}\bfseries}

\addtokomafont{chapter}{\fontfamily{phv}\fontsize{17}{19}\bfseries}
\addtokomafont{section}{\fontfamily{phv}\fontsize{13}{16}\bfseries}
\addtokomafont{subsection}{\fontfamily{phv}\fontsize{11}{14}\bfseries}
\addtokomafont{subsubsection}{\fontfamily{phv}\fontsize{9}{12}\bfseries}

% Größen der Beschriftungen vorgeben
\addtokomafont{caption}{\footnotesize}
\setkomafont{captionlabel}{\footnotesize}
% Größe der Kopf- und Fußzeile vorgeben
\setkomafont{pageheadfoot}{\footnotesize} 
% Größe der Seitenzahl
\setkomafont{pagenumber}{\normalsize}

% Farben im Dokument zulassen
\usepackage{color}
% Textfarbe schwarz definieren
\color[cmyk]{0,0,0,1}
% Bezeichnung für Überschriften 1cm vom linken Rand beginnen
\renewcommand*{\chapterformat}{\makebox[1.1cm][l]{\thechapter\autodot}}
\renewcommand*{\sectionformat}{\makebox[1.1cm][l]{\thesection\autodot}}
\renewcommand*{\subsectionformat}{\makebox[1.1cm][l]{\thesubsection\autodot}}
% Bezeichnungsnamen angeben
\addto\captionsngerman{\renewcommand{\figurename}{Abbildung}}
\addto\captionsngerman{\renewcommand{\tablename}{Tabelle}}

% Ansicht Literaturverzeichnis
%\bibliographystyle{plaindin}

% -------------------- Einstellungen weiterer Pakete ------------------------------
% Einbinden von Bildern ermöglichen
\usepackage{graphicx}	
% Gedrehte Objekte ermöglichen
\usepackage{rotating}
% Erweiterte Tabellenumgebung
\usepackage{tabularx}
% Erweiterte Flatersatz-Kommandos
\usepackage{ragged2e}
% Linksbündiger Flattersatz in den Bezeichnungen
%\usepackage[justification=RaggedRight]{caption}
\usepackage[justification=justified]{caption}
\captionsetup[subfigure]{justification=RaggedRight}
% Neuer Spaltentyp "L" mit Breitenangabe für linksbündigen Flattersatz
\newcolumntype{L}[1]{>{\RaggedRight\arraybackslash}p{#1}}
% Mathematische Symbole
\usepackage{amsmath,amssymb}
% Zeilen in Tabellen können verbunden werden
\usepackage{multirow}
% Zusätzliche Textsymbole zur Verfügung stellen
\usepackage{textcomp}
% Operatorensymbole definieren
\newcommand{\real}{\operatorname{Re}}				% Realteil
\newcommand{\opdiv}{\operatorname{div}}			% Divergenzoperator
\newcommand{\rot}{\operatorname{rot}}				% Rotationsoperator
\newcommand{\grad}{\operatorname{grad}}			% Gradientenoperator
\newcommand{\imag}{\operatorname{Im}}				% Imaginärteil
\newcommand{\imein}{\operatorname{j}}				% imaginäre Einheit j
% Darstellung urls mit Zeilenumbruch
\usepackage[hyphens]{url}
% Erweiterte Listenanweisungen
\usepackage{etoolbox}
% Zeilenumbrüche in urls nach folgenden Zeichen
\appto\UrlBreaks{\do\a\do\b\do\c\do\d\do\e\do\f\do\g\do\h\do\i\do\j\do\k\do\l\do\m\do\n\do\o\do\p\do\q\do\r\do\s\do\t\do\u\do\v\do\w\do\x\do\y\do\z\do\/\do\.}
% Einzug im Abbildungsverzeichnis zu Null setzen
\renewcommand{\cftfigindent}{0cm}
% Einzug im Tabellenverzeichnis zu Null setzen
\renewcommand{\cfttabindent}{0cm}
% Darstellung und Verlinkungen im pdf-Dokument einstellen
\usepackage[hidelinks,								% Links als normaler Text darstellen
	pdfpagemode = UseNone,							% Lesezeichen im pdf-Reader nicht anzeigen
	pdfpagelayout = TwoColumnRight,			% Seitenanzeige des pdf-Dokuments angeben
	pdfauthor = {\autor},								% Autor des pdf-Dokuments
	pdftitle = {\pdftitle}]							% Titel des pdf-Dokuments
	{hyperref}
% deutsches Abkürzungsverzeichnis erstellen
\usepackage[german]{nomencl}
% Befehl für einen Eintrag im Abkürzungsverzeichnis in "\sym" umbennen
\let\sym\nomenclature
% Name des Abkürzungsverzeichnis ändern
\renewcommand{\nomname}{Abkürzungs- und Symbolverzeichnis}
% Spaltenbreite der Formelzeichen auf "20 %" der Textbreite setzen
\setlength{\nomlabelwidth}{.2\textwidth}
% Einheiten in die Bezeichnung mit aufnehmen und rechtsbündig setzen
\newcommand{\nomunit}[1]{\renewcommand{\nomentryend}{\hspace*{\fill}#1}}
% Zeilenabstände verkleineren auf normalen Textabstand
\setlength\nomitemsep{-\parsep}
% Abkürzungsverzeichnis erzeugen
\makenomenclature
% Weitere Verzeichnisse erzeugen
\makeindex
% Blindtext zur Textdarstellung in der Vorlage ermöglichen
\usepackage{blindtext}


\AtBeginDocument{% 
  \newcaptionname{ngerman}\equationname{Formel}% 
  \newcaptionname{ngerman}\listequationname{Formelverzeichnis}% 
}

\DeclareNewTOC[ 
  indent=0pt,
  hang=2em,
  type=equation,
  name={Gl.}, 
  types=equations, 
  listname={Formelverzeichnis}, 
]{equ} 
\newcommand{\equationentry}[2][\theequation]{% 
  \addxcontentsline{equ}{equation}[{#1}]{\kern 1em #2}% 
} 
\BeforeStartingTOC[equ]{\def\autodot{:}} 
% Titel
\newcommand{\headtitle}{Titel der Dissertation}

%\makeindex

\newcommand{\nauthor}{Vorname Nachname}
\newcommand{\akadtitel}{M.Sc.}
\newcommand{\geburtsort}{Geburtsort}
%
\newcommand{\ntitle}{Titel der Dissertation}
%
\newcommand{\referent}{Prof. Dr.-Ing. Hauptreferent}
\newcommand{\korreferent}{Prof. Dr.-Ing. Korreferent}
\newcommand{\ndatum}{tt.mm.jjjj}

%\renewcommand{\figurename}{Abbildung}
%\renewcommand{\tablename}{Tabelle}

%\ohead[]{}
%\ofoot[\pagemark]{\pagemark} % Seitenzahlen unten au�en

%\ihead[]{}
%\ohead[]{\headmark} % Kopfzeilen oben au�en

%%%%%%%%%%%%%%%%%%%%%%%%%%%%%%%%%%%%%%%%%%%%%%%%%%

\begin{document}

\pagenumbering{gobble}
% Voreingestellter Seitenstil verwenden
\pagestyle{scrheadings}
% R�mische Seitennummerierung verwenden

% Seitenzahl mit 5 beginnen (KSP setzt 4 beschriftete Seiten davor)
%\setcounter{page}{5}

%%%%%%%%%%%%%%%%%%%%%%%%%%%%%%%%%%%%%%%%%%%%%%%%%%

% Titelblatt
\include{Inhalt/Titelblatt}

%%%%%%%%%%%%%%%%%%%%%%%%%%%%%%%%%%%%%%%%%%%%%%%%%%

\frontmatter

% Zusammenfassung
\include{Inhalt/Zusammenfassung}

% Vorwort
\include{Inhalt/Vorwort}

% Inhaltsverzeichnis
\markboth{Inhaltsverzeichnis}{Inhaltsverzeichnis}
\renewcommand{\contentsname}{Inhaltsverzeichnis}

\makeatletter
\renewcommand*\@pnumwidth{2em} % für dreistelle Seitenzahlen im InhaltsVZ innerhalb des druckbaren Bereichs
\makeatother

\BeforeStartingTOC{\setstretch{1.075}} % eventuell ändern um eine schönere Seitendarstellung zu erhalten
\tableofcontents

%%%%%%%%%%%%%%%%%%%%%%%%%%%%%%%%%%%%%%%%%%%%%%%%%%


% Abk�rzungen und Symbole
\markboth{Abk\"urzungen und Symbole}{Abk\"urzungen und Symbole}
\chapter{Abk\"urzungen und Symbole}
\label{cha:Abkuerzungen_Symbole}

%%%%%%%%%%%%%%%%%%%%%%%%%%%%%%%%%%%%%%%%%%%%%%%%%%

\section*{Abk\"urzungen}
\markboth{Abk\"urzungen und Symbole}{Abk\"urzungen und Symbole}
%\begin{spacing}{0.9}	% eventuell ändern um eine schönere Seitendarstellung zu erhalten
\begin{acronym}[XXXXXXX]
\setlength{\itemsep}{0.0585cm}
\begingroup
\acro{IHE}{Institut f\"ur Hochfrequenztechnik und Elektronik}
\acro{KIT}{Karlsruher Institut f\"ur Technologie}
\endgroup
\end{acronym}
%\end{spacing}

%%%%%%%%%%%%%%%%%%%%%%%%%%%%%%%%%%%%%%%%%%%%%%%%%%

\section*{Konstanten}
\begin{description}[leftmargin=\widthof{\textbf{XXXXXXX}\hspace{\labelsep}},style=nextline]
\item[$\pi$] Kreiszahl Pi: $3$,$14159\ldots$
\end{description}

%%%%%%%%%%%%%%%%%%%%%%%%%%%%%%%%%%%%%%%%%%%%%%%%%%

\section*{Lateinische Symbole und Variablen}
\subsection*{Kleinbuchstaben}
\begin{description}[leftmargin=\widthof{\textbf{XXXXXXX}\hspace{\labelsep}},style=nextline]
\item[$f$] Frequenz
\item[$g$] Gewichtungsfaktor
\end{description}

\subsection*{Gro{\ss}buchstaben}
\begin{description}[leftmargin=\widthof{\textbf{XXXXXXX}\hspace{\labelsep}},style=nextline]
\item[$B$] Bandbreite
\item[$\mathbf{B}$] Strahlformungsmatrix
\end{description}

%%%%%%%%%%%%%%%%%%%%%%%%%%%%%%%%%%%%%%%%%%%%%%%%%%

\clearpage

\section*{Griechische Symbole und Variablen}
\begin{description}[leftmargin=\widthof{\textbf{XXXXXXX}\hspace{\labelsep}},style=nextline]
\item[$\lambda$] Wellenl\"ange
\item[$\varphi$] Phasenverschiebung
\end{description}
%%%%%%%%%%%%%%%%%%%%%%%%%%%%%%%%%%%%%%%%%%%%%%%%%%

\section*{Operatoren und mathematische Symbole}
\begin{description}[leftmargin=\widthof{\textbf{XXXXXXX}\hspace{\labelsep}},style=nextline]
\item[$a$] komplexe Gr\"o{\ss}e
\item[$\vec a$] komplexer Vektor
%& && & &&
\end{description}

%%%%%%%%%%%%%%%%%%%%%%%%%%%%%%%%%%%%%%%%%%%%%%%%%%

\section*{Allgemeine Tiefindizes}
\begin{description}[leftmargin=\widthof{\textbf{XXXXXXX}\hspace{\labelsep}},style=nextline]
\item[$\mathrm{adapt}$] adaptiv
\item[$\mathrm{inkoh}$] inkoh\"arent
\end{description}

\cleardoublepage % jedes neue Kapitel beginnt auf der rechten Seite

%%%%%%%%%%%%%%%%%%%%%%%%%%%%%%%%%%%%%%%%%%%%%%%%%%

\pagenumbering{roman}
\setcounter{page}{1}

% Haupteil
\mainmatter

\renewcommand{\arraystretch}{1.2} % Zeilenabstand innerhalb Tabellen (urspruenglich 1,4)
\setlength{\tabcolsep}{1mm} % Abstand zwischen Spalten in Tabellen

\setlist[itemize]{itemsep=0mm, topsep=1mm} % Abst�nde bei Aufz�hlungspunkten

\captionsetup[table]{belowskip=1mm} % gr��erer Abstand zwischen Caption und Tabelle

\pagenumbering{arabic} % Arabische Seitennummerierung verwenden

%\RedeclareSectionCommand[beforeskip=-1.00\baselineskip,afterskip=0.50\baselineskip]{section}
%\RedeclareSectionCommand[beforeskip=-0.75\baselineskip,afterskip=0.50\baselineskip]{subsection}
%\RedeclareSectionCommand[beforeskip=-0.50\baselineskip,afterskip=0.25\baselineskip]{subsubsection}

% Einleitung
\markboth{Einleitung}{Einleitung}
\chapter{Einleitung}
\label{cha:Einleitung}
Lorem ipsum dolor sit amet, consetetur sadipscing elitr, sed diam nonumy eirmod tempor invidunt ut labore et dolore magna aliquyam erat, sed diam voluptua. At vero eos et accusam et justo duo dolores et ea rebum. Stet clita kasd gubergren, no sea takimata sanctus est Lorem ipsum dolor sit amet. Lorem ipsum dolor sit amet, consetetur sadipscing elitr, sed diam nonumy eirmod tempor invidunt ut labore et dolore magna aliquyam erat, sed diam voluptua. At vero eos et accusam et justo duo dolores et ea rebum. Stet clita kasd gubergren, no sea takimata sanctus est Lorem ipsum dolor sit amet.

\section{Motivation und Umfeld der Arbeit}
\label{sec:Motivation_und_Umfeld}
Lorem ipsum dolor sit amet, consetetur sadipscing elitr, sed diam nonumy eirmod tempor invidunt ut labore et dolore magna aliquyam erat, sed diam voluptua. At vero eos et accusam et justo duo dolores et ea rebum. Stet clita kasd gubergren, no sea takimata sanctus est Lorem ipsum dolor sit amet. Lorem ipsum dolor sit amet, consetetur sadipscing elitr, sed diam nonumy eirmod tempor invidunt ut labore et dolore magna aliquyam erat, sed diam voluptua. At vero eos et accusam et justo duo dolores et ea rebum. Stet clita kasd gubergren, no sea takimata sanctus est Lorem ipsum dolor sit amet.

\section{Stand der Technik}
\label{sec:Stand_der_Technik}
Lorem ipsum dolor sit amet, consetetur sadipscing elitr, sed diam nonumy eirmod tempor invidunt ut labore et dolore magna aliquyam erat, sed diam voluptua. At vero eos et accusam et justo duo dolores et ea rebum. Stet clita kasd gubergren, no sea takimata sanctus est Lorem ipsum dolor sit amet. Lorem ipsum dolor sit amet, consetetur sadipscing elitr, sed diam nonumy eirmod tempor invidunt ut labore et dolore magna aliquyam erat, sed diam voluptua. At vero eos et accusam et justo duo dolores et ea rebum. Stet clita kasd gubergren, no sea takimata sanctus est Lorem ipsum dolor sit amet. Lorem ipsum dolor sit amet, consetetur sadipscing elitr, sed diam nonumy eirmod tempor invidunt ut labore et dolore magna aliquyam erat, sed diam voluptua:
\begin{itemize}
\item[$\bullet$] Bulletpoint 1
\item[$\bullet$] Bulletpoint 2
\item[$\bullet$] Bulletpoint 3
\item[$\bullet$] Bulletpoint 4
\item[$\bullet$] Bulletpoint 5
\end{itemize}

Lorem ipsum dolor sit amet, consetetur sadipscing elitr, sed diam nonumy eirmod tempor invidunt ut labore et dolore magna aliquyam erat, sed diam voluptua. At vero eos et accusam et justo duo dolores et ea rebum. Stet clita kasd gubergren, no sea takimata sanctus est Lorem ipsum dolor sit amet. Lorem ipsum dolor sit amet, consetetur sadipscing elitr, sed diam nonumy eirmod tempor invidunt ut labore et dolore magna aliquyam erat, sed diam voluptua. At vero eos et accusam et justo duo dolores et ea rebum. Stet clita kasd gubergren, no sea takimata sanctus est Lorem ipsum dolor sit amet.

Lorem ipsum dolor sit amet, consetetur sadipscing elitr, sed diam nonumy eirmod tempor invidunt ut labore et dolore magna aliquyam erat, sed diam voluptua. At vero eos et accusam et justo duo dolores et ea rebum. Stet clita kasd gubergren, no sea takimata sanctus est Lorem ipsum dolor sit amet. Lorem ipsum dolor sit amet, consetetur sadipscing elitr, sed diam nonumy eirmod tempor invidunt ut labore et dolore magna aliquyam erat, sed diam voluptua.

% Grundlagen
\include{Inhalt/02_Grundlagen}

% Schlussfolgerungen und Ausblick
\renewcommand{\chaptermark}[1]{\markboth{\thechapter\ \  #1}{\thechapter\ \  #1}}	% Setzt Kapitel�berschrift auch auf rechte Seite der Kopfzeile, da keine Unterkapitel
\include{Inhalt/07_Schlussfolgerungen_und_Ausblick}

%%%%%%%%%%%%%%%%%%%%%%%%%%%%%%%%%%%%%%%%%%%%%%%%%%

\appendix

\include{Inhalt/Anhang}

% Literaturverzeichnis
\renewcommand{\chaptermark}[1]{\markboth{\thechapter\ \  #1}{\thechapter\ \  #1}}	% Setzt Kapitel�berschrift auch auf rechte Seite der Kopfzeile, da keine Unterkapitel
\setlength{\parskip}{0.3\baselineskip plus 0.15\baselineskip minus 0.15\baselineskip} % �nderung der Abst�nde zwischen den Eintr�gen
\markboth{Literaturverzeichnis}{Literaturverzeichnis}
\chapter*{Literaturverzeichnis}
\addcontentsline{toc}{chapter}{Literaturverzeichnis}
\renewcommand{\refname}{Literaturverzeichnis}
\renewcommand\bibname{Literaturverzeichnis}
\begingroup % Überschrift von \bibliography{} löschen
\renewcommand{\chapter}[2]{}
\renewcommand{\section}[2]{}
\nocite{*}
\bibliographystyle{alpha}
\bibliography{Literatur/Externe_Literatur}{}
\endgroup

%%%%%%%%%%%%%%%%%%%%%%%%%%%%%%%%%%%%%%%%%%%%%%%%%%

\chapter*{Eigene Ver\"offentlichungen}
\addcontentsline{toc}{chapter}{Eigene Ver\"offentlichungen}

\section*{Journalartikel}
\addcontentsline{toc}{section}{Journalartikel}
\renewcommand{\refnamejournal}{Journalartikel}
\begingroup % Überschrift von \bibliography{} löschen
\renewcommand{\chapter}[2]{}
\renewcommand{\section}[2]{}
\nocitejournal{*}
\bibliographystylejournal{plain}
\bibliographyjournal{Literatur/Eigene_Journal_Papers}{}
\endgroup

%%%%%%%%%%%%%%%%%%%%%%%%%%%%%%%%%%%%%%%%%%%%%%%%%%

\section*{Konferenzbeitr\"age}
\addcontentsline{toc}{section}{Konferenzbeitr\"age}
\renewcommand{\refnameconference}{Konferenzbeitr\"age}
\begingroup % Überschrift von \bibliography{} löschen
\renewcommand{\chapter}[2]{}
\renewcommand{\section}[2]{}
\nociteconference{*}
\bibliographystyleconference{plain}
\bibliographyconference{Literatur/Eigene_Konferenz_Papers}{}
\endgroup

%%%%%%%%%%%%%%%%%%%%%%%%%%%%%%%%%%%%%%%%%%%%%%%%%%

%\markboth{Literaturverzeichnis}{Literaturverzeichnis}
%\renewcommand{\refname}{Literaturverzeichnis}
%\renewcommand\bibname{Literaturverzeichnis}
%\bibliography{Literatur/Eigene_Konferenz_Papers}{}
%\bibliographystyle{alpha}
%\nocite{*}

% Lebenslauf f�r einzureichende Version
%\cleardoublepage
%\include{content/lebenslauf}

\end{document}